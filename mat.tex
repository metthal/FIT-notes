\documentclass[12pt]{article}

\usepackage[margin=1in]{geometry} 
\usepackage{amsmath,amsthm,amssymb,amsfonts}
\usepackage[utf8]{inputenc}
\usepackage[slovak]{babel}

\newcommand{\pipesep}{\hspace{3pt} \vert \hspace{3pt}}
\setlength\parindent{0pt}

\begin{document}

\title{Poznamky z MATu}
\author{Marek Milkovic}
\date{}
\maketitle

\section{Vyrokova logika}

Vyrokova logika skuma sposob tvorby zlozenych vyrokov z danych jednoduchych vyrokov a zavislosti
pravdivosti zlozeneho vyroku na pravdivosti vyrokov z ktorych je zlozeny.
\\
$P$ - neprazdna mnozina symbolov - \emph{prvotnych vyrokov} \\
Prvotne vyroky predstavuju jednoduche vyroky. Zlozene vyroky sa skladaju z jednoduchych pomocou
logickych spojek - $\neg, \land, \lor, \to, \leftrightarrow$. \\
$L_{P}$ - jazyk vyrokovej logiky - prvky $P$, logicke spojky a zatvorky $()$. \\
Ulohu zlozenych vyrokov hraju \emph{vyrokove formule} jazyka $L_{P}$, ktore vznikaju ako:
\begin{enumerate}
		\item Kazda prvotna formula $p \in P$ je vyrokova formula
		\item Ak su $A,B$ vyrokove formule, tak aj $\neg A$, $A \land B$, $A \lor B$,
			$A \to B$, $A \leftrightarrow B$.
		\item Kazda vyrokova formula vznikne konecnym poctom pouzitia pravidiel 1 a 2
\end{enumerate}
\vspace{1cm}
Pravdivostne ohodnotenie prvotnych formuli - $v: P \to \{0,1\}$ \\
Rozsirenie na vsetky formule - $\bar{v}$
\begin{enumerate}
		\item $\bar{v}(p) = v(p) \forall p \in P$
		\item Ak su $A,B$ vyrokove formule, tak $\bar{v}(\neg A), \bar{v}(A \land B),
			\bar{v}(A \lor B), \bar{v}(A \to B), \bar{v}(A \leftrightarrow B)$ sa definuje pomocou
			pravdivostnej tabulky v zavislosti na hodnote $\bar{v}(A), \bar{v}(B)$
			(a tie pozname velmi dobre)
\end{enumerate}
Vyrokova formula $A$ je pravdiva pri ohodnoteni $v$ - $\bar{v}(A) = 1$ \\
Vyrokova formula $A$ je tautologia ak $\bar{v}(A) = 1$ pre lubovolne ohodnotenie $v$ - $\models A$\\
Vyrokove formule su \emph{ekvivalentne} prave vtedy ak $\bar{v}(A) = \bar{v}(A)$ pre lubovolne
	ohodnotenie $v$ - $A \leftrightarrow B$ musi byt tautologia \\
Kazda vyrokova formula je logicky ekvivalentna niektorej vyrokovej formuli, v ktorej su len spojky
	$\neg, \to$ (ale aj $\neg, \land$ a $\neg, \lor$) \\
\begin{enumerate}
\item Nicodova spojka - $\downarrow$ - NOR
\item Shefferova spojka - $\vert$ - NAND
\end{enumerate}
\subsection{Dokazatelnost vo vyrokovej logike}
Vyrokova logika bude budovana ako formalna axiomaticka teoria \\
\textbf{Abeceda}
\begin{enumerate}
	\item mnozina $P$
	\item logicke spojky $\neg, \to$
	\item pomocne symboly $()$
\end{enumerate}
\textbf{Formule}
\begin{enumerate}
	\item vsetky prvotne formule su formule
	\item ak su $A$, $B$ formule tak aj $\neg A$ a $A \to B$ su formule
	\item opakovanim 1 a 2 vznikaju formule
\end{enumerate}
\textbf{Jazyk} - abeceda a formule tvoria jazyk \\
\textbf{Axiomy} - nekonecne mnoho axiomu zadanych pomocou 3 schemat
\begin{itemize}
	\item (A1) $A \to (B \to A)$
	\item (A2) $(A \to (B \to C)) \to ((A \to B) \to (A \to C))$
	\item (A3) $(\neg B \to \neg A) \to (A \to B)$
\end{itemize}
\textbf{Odvodzovacie pravidlo} - \emph{Modus ponens} - pravidlo odlucenia -
	z formuli $A, (A \to B)$ sa odvodi $B$. $A, (A \to B)$ su predpoklady, $B$ je zaver \\
\textbf{Dokaz} - lubovolna konecna postupnost $A_{1}, ..., A_{n}$ vyrokovych formuli takych,
	ze $\forall i \le n$ je $A_{i}$ bud axiom, alebo zaver pravidla modus ponens \\
- Formula $A$ je \emph{dokazatelna}, ak existuje dokaz, ktoreho poslednou formulou je $A$
- $\vdash A$ \\
- Kazda dokazatelna formula vo vyrokovej logike je tautologia \\
\vspace{0.5cm}
Nech $T$ je mnozina formuli vyrokovej logiky. Hovorime, ze konecna postupnost $A_{1},...,A_{n}$
je \emph{dokazom formule A z predpokladu T}, ak $A_{n}$ je formula $A$ a pre lubovolne
$i \le n$ plati ze $A_{i}$ je bud axiom, alebo formula z $T$ alebo $A_{i}$ je zaverom
modus ponens - formula $A$ je \emph{dokazatelna z predpoklatu T} - $T \vdash A$ \\
\textbf{Veta o dedukcii} - Nech $T$ je mnozina formuli, nech $A,B$ su formule, potom
	$T \vdash A \to B$ prave vtedy ked $T \cup \{A\} \vdash B$. \\
\textbf{Veta o uplnosti} - pre lubovolnu formulu $A$ vo vyrokovej logike plati $\vdash A$,
prave ked $\models A$.

\section{Predikatova logika 1.radu}
\textbf{Premenne} - prvky z nejakej mnoziny \\
\textbf{Funkcne symboly} - $f,g,h...$ - operacie, $n$-arny symbol \\
\textbf{Predikat} - vztah medzi urcitym poctom objektov \\
\textbf{Predikatovy symbol} - $p,q,r...$ - vyjadrujeme nimi predikaty - $n$-arny symbol \\
\textbf{Atomicka formula} - zlozena premennych, konstant, funkcnych symbolov
	a predikatovych symbolov\\
\textbf{Logicke spojky} - rovnake ako vo vyrokovej logike\\
\textbf{Kvantifikatory premennych} - univerzalny $\forall$ a existencny $\exists$\\
\textbf{Abeceda jazyka predikatovej logiky} - premenne, konstanty, funkcne a predikatove symboly,
	logicke spojky, kvantifikatory a pomocne symboly $()$\\
Premenna predikatovej logiky 1. radu predstavuje konkretny objekt, nedokaze predstavovat
inu mnozinu \\

\textbf{Term}
\begin{enumerate}
	\item Kazda premenna je term
	\item Ak je $f$ $n$-arny funkcny symbol, a $t_{1},...,t_{n}$ su termy, tak aj
		$f(t_{1},...,t_{n})$ je term
	\item Konecny uzitim pravidiel 1 a 2 vznikne term
\end{enumerate}

\textbf{Atomicka formula}\\
Ak je $p$ $n$-arny predikatovy symbol a $t_{1},...,t_{n}$ su termy, tak $p(t_{1},...,t_{n})$
je \emph{atomicka formula} \\

\textbf{Formule} \\
\begin{enumerate}
	\item Kazda atomicka formula je formula
	\item Ak su $\varphi, \psi$ formule, tak aj $\neg \varphi, \varphi \land \psi, \varphi \lor \psi,
		\varphi \to \psi, \varphi \leftrightarrow \psi$ su formule
	\item Ak je $x$ premenna a $\varphi$ formula, tak $\forall x \varphi$ a $\exists x \varphi$
		su formule
	\item Kazda formula vznikne konecnym uzitim formuli 1-3
\end{enumerate}
Vyskyt premennej $x$ vo formuli $\varphi$ sa nazyva viazany, ak sa nachadza v nejakej podformuli
v tvare $\forall x \psi$ alebo $\exists x \psi$. V opacnom pripade sa jedna o vyskyt volny.
Formula co neobsahuje ziadnu volnu premennu sa nazyva \emph{uzavrena formula} alebo aj \emph{vyrok}.
\subsection{Semantika predikatovej logiky}
Chceme dat intepretaciu symbolom jazyka predikatovej logiky 1. radu. Vymedzime teda obor,
ktory bude urcovat mozne hodnoty premennych $M$. Funkcnym symbolom budu odpovedat operacie na $M$.
Predikatovym symbolom budu odpovedat vztahy medzi objektami $M$, ktore je mozne popisat ako
relacie na $M$. \\
\\
\textbf{Realizacia jazyka} - Nech je $L$ jazyk 1. radu. Realizaciou jazyka rozumieme algebraicku
strukturu $\mathcal{M}$, ktora sa sklada z
\begin{enumerate}
	\item neprazdnej mnoziny $M$, ktoru nazveme \emph{univerzum}
	\item pre kazdy funkcny symbol $f$ pocetnosti $n$ je dane zobrazenie
		$f_{\mathcal{M}}: M^{n} \to M$
	\item pre kazdy predikatovy symbol $p$ pocetnosti $n$, okrem rovnosti, je dana relacia
		$p_{\mathcal{M}} \subseteq M^{n}$
\end{enumerate}
\textbf{Ohodnotenie premennych} - Lubovolne zobrazenie $e$ mnoziny vsetkych premennych do
univerza $M$ dane realizaciou $\mathcal{M}$ jazyka $L$ \\
- Ak je $x$ premenna, $e$ ohodnotenie premennych, $m \in M$, potom znacime $e(x/m)$ \\
- \textbf{Hodnotu termu} $t$ v realizacii $\mathcal{M}$ pri danom ohodnoteni $e$ znacime
	$t[e]$ a definovana je nasledovne
	\begin{enumerate}
		\item Ak je $t$ premenna $x$ potom $t[e]$ je $e(x)$
		\item Ak je $t$ v tvare $f(t_{1},...,t_{n})$, kde $f$ je $n$-arny funkcny symbol
			a $t_{i}$ su termy, potom $t[e]$ je $f_{\mathcal{M}}(t_{1}[e],...,t_{n}[e])$
	\end{enumerate}
\vspace{0.5cm}
Nech $\mathcal{M}$ je realizacia jazyka $L$, nech $e$ je ohodnotenie premennych a $\varphi$
je formula jazyka $L$. Definujeme, co znamena \emph{formula $\varphi$ je pravidva v
$\mathcal{M}$ pri ohodnoteni e} - $\mathcal{M} \models \varphi[e]$
\begin{enumerate}
	\item Ak je $\varphi$ atomicka formula v tvare $p(t_{1},...,t_{n})$, tak
		$\mathcal{M} \models \varphi[e]$ prave ked $(t_{1}[e],...,t_{n}[e]) \in p_{\mathcal{M}}$
	\item Ak je $\varphi$ atomicka formula v tvare $t_{1} = t_{2}$, tak
		$\mathcal{M} \models \varphi[e]$ prave ked $t_{1}[e]$ je ten isty prvok ako $t_{2}[e]$
	\item Ak je $\varphi$ v tvare $\neg \psi$, tak 
		$\mathcal{M} \models \varphi[e]$ prave ked $\mathcal{M} \not\models \psi[e]$
	\item Ak je $\varphi$ v tvare $\eta \land \psi$, $\eta \lor \psi$, $\eta \to \psi$,
		$\eta \leftrightarrow \psi$ tak $\mathcal{M} \models (\eta \land \psi)[e]$ prave vtedy ak
		$\mathcal{M} \models \eta[e]$ a zaroven $\mathcal{M} \models \psi[e]$ a analogicky
		pre zvysok
	\item Ak je $\varphi$ v tvare $\forall x \psi$, tak
		$\mathcal{M} \models \varphi[e]$ prave vtedy ak
		$\forall m \in M : \mathcal{M} \models \psi[e(x/m)]$
	\item Ak je $\varphi$ v tvare $\exists x \psi$, tak
		$\mathcal{M} \models \varphi[e]$ prave vtedy ak
		$\exists m \in M : \mathcal{M} \models \psi[e(x/m)]$
\end{enumerate}
Formula $\varphi$ je \emph{splnena} v realizacii $\mathcal{M}$, ak je $\varphi$ pravdiva
v $\mathcal{M}$ pri kazdom ohodnoteni $e$. Potom piseme $\mathcal{M} \models \varphi$.

Ak je $\varphi$ uzavrena formula, tak vravime ze $\varphi$ je \emph{pravdiva} v $\mathcal{M}$.

Formula sa nazyva \emph{splnitelna}, ak je splnena v nejakej realizacii.

Formula je \emph{logicky platna}, ak je splnena v kazdej realizacii jazyka $L$ - $\models \varphi$

Formula $\varphi, \psi$ su \emph{logicky ekvivalentne} ak v lubovolnej realizacii $\mathcal{M}$
a pri lubovolnom ohodnoteni $e$ je $\mathcal{M} \models \varphi[e]$ prave vtedy ak
$\mathcal{M} \models \psi[e]$.

\subsection{Substitucia termov za premenne}
Ak je $\varphi$ formula, $x$ je premenna a $t$ je term, potom vyraz ktory vznikne z formule
$\varphi$ nahradenim kazdeho volneho vyskytu premennej $x$ termom $t$ je opat formula.

Term $t$ je \emph{substituovatelny} za $x$ do formule $\varphi$, ak ziadny volny vyskyt premennej
$x$ sa nenachadza v obore kvatifikacie nejakeho kvantifikatoru premennej $y$, kde $y$ je
premenna obsiahnuta v terme $t$.

Budeme znacit $\varphi_{x}[t]$ formulu, ktora vznikne z $\varphi$ nahradenim kazdeho volneho
vyskytu $x$ termom $t$.

Ak mame $\varphi, x, t$ a $t$ je substituovatelny za $x$ do $\varphi$, potom
$(\forall x \varphi) \to \varphi_{x}[t]; \varphi_{x}[t] \to (\exists x \varphi)$ su logicky
platne formule.

\subsection{Formalny system predikatovej logiky}
Zavedieme predikatovu logiku ako formalny axiomaticky system. Obmedzime sa len na logicke spojky
$\neg, \to$ a kvantifikatoru $\forall$.

\textbf{Schema vyrokovych axiomov}
\begin{itemize}
	\item $\varphi \to (\psi \to \varphi)$
	\item $(\varphi \to (\psi \to \eta)) \to ((\varphi \to \psi) \to (\varphi \to \eta))$
	\item $(\neg\psi \to \neg\varphi) \to (\varphi \to \psi)$
\end{itemize}
\textbf{Schema axiomu kvantifikatoru}
\begin{itemize}
	\item $(\forall x(\varphi \to \psi)) \to (\varphi \to (\forall x \psi))$
		($x$ nema volny vyskyt v $\varphi$)
\end{itemize}
\textbf{Schema axiomu substitucie}
$t$ je substituovatelny za $x$ do $\varphi$
\begin{itemize}
	\item $(\forall x \varphi) \to \varphi_{x}[t]$
\end{itemize}
\textbf{Schema axiomov rovnosti}
\begin{itemize}
	\item $x = x$
	\item $(x_{1} = y_{1} \to (x_{2} = y_{2} \to ( ... (x_{n} = y_{n} \to
		f(x_{1}, x_{2}, ..., x_{n}) = f(y_{1}, y_{2}, ..., y_{n}))...)))$
	\item $(x_{1} = y_{1} \to (x_{2} = y_{2} \to ( ... (x_{n} = y_{n} \to
		p(x_{1}, x_{2}, ..., x_{n}) \to p(y_{1}, y_{2}, ..., y_{n}))...)))$
\end{itemize}
\subsection{Odvodzovacie pravidla predikatovej logiky}
\begin{itemize}
	\item \textbf{Pravidlo odlucenia - modus ponens} - Z formuli $\varphi$,$\varphi \to \psi$
		sa odvodi formula $\psi$
	\item \textbf{Pravidlo zobecnenia - generalizacia} - Pre lubovolnu premennu $x$ sa z formule
		$\varphi$ odvodi formula $(\forall x \varphi)$
\end{itemize}

\textbf{Dokaz} - lubovolna postupnost $\varphi_{1},...,\varphi_{n}$ formuli jazyka $L$ v ktorej
pre kazdy index $i$ je formula $\varphi_{i}$ bud axiom predikatovej logiky alebo je ju mozne
odvodit z niektorych predchadzajucich formuli pouzitim pravidla odlucenia alebo zobecnenia.

Formula $\varphi$ je \emph{dokazatelna}, pokial existuje dokaz, ktoreho poslednou formulou je
$\varphi$. Piseme $\vdash \varphi$.

Bud $T$ mnozina formuli jazyka $L$. Formula $\varphi$ je \emph{dokazatelna z predpokladov T}, ak
existuje dokaz z predpokladu $T$, teda postupnost formuli, ze jej poslednou formulou je $\varphi$
a kazda predosla formula je bud axiom, alebo je to formula z $T$ alebo vznikla uzitim pravidla
odlucenia alebo zobecnenia. Piseme $T \vdash \varphi$.

Lubovolna formula jazyka $L$ dokazatelna v predikatovej logike 1. radu je logicky platnou tj.
splnena v kazdej realizacii jazyka $L$.

\textbf{Lemma} - Ak $\vdash \varphi \to \psi$ a $x$ nema volny vyskyt v $\varphi$, tak
$\vdash \varphi \to (\forall x \psi)$

\textbf{Lemma} - Ak $\vdash \varphi \to \psi$ a $x$ nema volny vyskyt v $\varphi$, tak
$\vdash (\exists x \varphi) \to \psi$

\textbf{Lemma} - Ak $\varphi$ je formula, $x$ je premenna a $t$ term substituovatelny za $x$
do $\varphi$, tak $\vdash \varphi_{x}[t] \to (\exists x \varphi)$

\textbf{Lemma} - Nech $\varphi'$ je instanciou formule ($\varphi'$ je teda v tvare
$\varphi_{x_{1},...,x_{n}}[t_{1},...,t_{n}])$. Ak $\vdash \varphi$, tak aj $\vdash \varphi'$.

Ak su $x_{1},...,x_{n}$ vsetky volne premenne vo formuli $\varphi$, potom formulu
$\forall x_{1},...,x_{n} \varphi$ nazyvame \emph{uzaverom formule $\varphi$}.

\textbf{Veta o uzavere} - Ak je $T$ mnozina formuli a $\varphi'$ uzaver formule $\varphi$,
potom $T \vdash \varphi$ prave vtedy ked $T \vdash \varphi'$

\textbf{Veta o dedukcii} - Nech $T$ je mnozina formuli jazyka $L$, nech $\varphi$ je uzavrena
formula, $\psi$ je lubovolna formula jazyka $L$. Potom $T \vdash \varphi \to \psi$ prave vtedy
ked $T \cup \varphi \vdash \psi$.

\textbf{Veta o konstantach} - Nech $T$ je mnozina formuli jazyka $L$, nech $\varphi$ je formula.
Nech $x_{1},...,x_{n}$ su premenne a nech $c_{1},...,c_{n}$ su nove konstanty, ktorych priradenim
k $L$ vznikne jazyk $L'$. Potom $T \vdash \varphi_{x_{1},...,x_{n}}[c_{1},...,c_{n}]$, prave ked
$T \vdash \varphi$.

\textbf{Lemma} - Nech je jazyk $L$ s rovnostou
\begin{align*}
	\vdash x &= y \to y = x \\
	\vdash x = y &\to (y = z \to x = z)
\end{align*}

\textbf{Lemma} - Ak je $f$ funkcny symbol s pocetnostou $n$, ak je $p$ predikatovy symbol
s cetnostou $m$ jazyka $L$ a ak su $u,v,w,s_{1},...,s_{n},t_{1},...,t_{n}$ termy jazyka $L$.
\begin{enumerate}
	\item $\vdash u = u$
	\item $\vdash u = v \to v = u$
	\item $\vdash u = v \to (v = w \to u = w)$
	\item $\vdash s_{1} = t_{1} \to (s_{2} = t_{2} \to ... (s_{n} = t_{n} \to
		f(s_{1},...,s_{n}) = f(t_{1},...,t_{n}))...))$
	\item $\vdash s_{1} = t_{1} \to (s_{2} = t_{2} \to ... (s_{n} = t_{n} \to
		p(s_{1},...,s_{n}) \to p(t_{1},...,t_{n}))...))$
\end{enumerate}

\subsection{Prenexny tvar formuli}
Formula $A$ je v prenexnom tvare, pokial ma tvar $Q_{1}x_{1}...Q_{n}x_{n}B$, kde
\begin{enumerate}
	\item $n \ge 0$ pre kazde $i = 1,...,n$ je $Q_{i}$ bud $\forall$ alebo $\exists$
	\item $x_{1},...,x_{n}$ su navzajom rozne premenne
	\item $B$ je otvorena formula (neobsahuje kvantifikatory)
\end{enumerate}
\textbf{Prevod formuli do prenexneho tvaru}
\begin{enumerate}
	\item \textbf{Vylucenie zbytocnych kvantifikatorov} - vynechame kvantifikatory $\forall x$ a
		$\exists x$ v podformulach tvaru $\forall x B$ alebo $\exists x B$, pokial sa premenna
		$x$ nevyskytuje volne v $B$
	\item \textbf{Premenovanie premennych} - Postupujeme z lava do prava premenovavanim premennych
		v podformuliach
	\item \textbf{Eliminacia spojky $\leftrightarrow$} - Prevedieme pomocou
		\begin{equation*}
			A \leftrightarrow B ... (A \to B) \land (B \to A)
		\end{equation*}
	\item \textbf{Presun negacie dovnutra} - vykonavame postupne nahrady podformuli podla schemat
		\begin{flalign*}
			\neg(\forall x A) ... (\exists x \neg A) & \\
			\neg(\exists x A) ... (\forall x \neg A) & \\
			\neg(A \to B) ... (A \land \neg B) & \\
			\neg(A \lor B) ... (\neg A \land \neg B) & \\
			\neg(A \land B) ... (\neg A \lor \neg B) & \\
			\neg(\neg A) ... A &
		\end{flalign*}
	\item \textbf{Presun kvantifikatov dolava} - pre podformulu $B$, v ktorej sa nevyskytuje
		premenna $x$ vykonavame nahrady podla schemat
		\begin{flalign*}
			(QxA) \lor B ... Qx(A \lor B) & \\
			(QxA) \land B ... Qx(A \land B) & \\
			(QxA) \to B ... \bar{Q}x(A \to B) & \\
			A \to (QxB) ... Qx(A \to B) &
		\end{flalign*}
\end{enumerate}
\subsection{Veta o uplnosti}
Ak je jazyk $L$ 1.radu a $T$ mnozina formuli jazyka $L$, hovorime, ze $T$ je
	\emph{teoria 1. radu} s jazykom $L$.

Hovorime ze teoria $T$ je \emph{sporna}, pokial pre kazdu formulu $\varphi$ jazyka $L$ plati
	$T \vdash \varphi$. V opacnom pripade je \emph{bezesporna}.

Nech $T$ je mnozina formuli a $\varphi'$ je uzaver formule $\varphi$. Potom $T \vdash \varphi$
prave vtedy, ked $T \cup \{\neg\varphi'\}$ je sporna teoria.

Bud $T$ teoria s jazykom $L$ a nech $\mathcal{M}$ je nejaka realizacia jazyka $L$.
Hovorime ze $\mathcal{M}$ je \emph{model teorie T}, pokial $\mathcal{M} \models \varphi$
pre kazdu formulu $\varphi \in T$. Potom piseme $\mathcal{M} \models T$.

Formula $\varphi$ je \emph{dosledkom toerie T}, pokial pre kazdy model $\mathcal{M}$
teorie $T$ je $\mathcal{M} \models \varphi$. Piseme $T \models \varphi$.

\textbf{Veta o korektnosti} - Ak je $T$ teoria s jazykom $L$ a $\varphi$ formula, taka, ze
$T \vdash \varphi$, potom $T \models \varphi$.

\section{Algebraicke struktury}

Bud $A$ mnozina $n \in \mathbb{N}_{0}$. Zobrazenie $\omega: A^{n} \to A$ sa nazyva
\emph{n-arna operacia} na $A$.
\begin{equation*}
	\omega :
	\begin{cases}
		A^{n} \to A \\
		(x_{1},...,x_{n}) \mapsto \omega x_{1},...,x_{n}
	\end{cases}
\end{equation*}
Pre $n = 0$:
\begin{equation*}
	\omega :
	\begin{cases}
		A^{0} = \{\varnothing\} \to A \\
		\varnothing \mapsto \omega \varnothing = \omega
	\end{cases}
\end{equation*}
Binarne operacie ($n = 2$) znacime zvycajne nejakym symbolom $\circ$.
\begin{equation*}
	\circ :
	\begin{cases}
		A^{2} \to A \\
		(x,y) \mapsto x \circ y
	\end{cases}
\end{equation*}
Unarne operacie ($n = 1$)
\begin{equation*}
	\omega :
	\begin{cases}
		A \to A \\
		x \mapsto \omega x
	\end{cases}
\end{equation*}

Bud $A$ mnozina $n \in \mathbb{N}_{0}, D \subseteq A^{n}$. Potom zobrazenie
$\omega: D \to A$ sa nazyva \emph{n-arna parcialna operacia} na $A$. ($-,/$ na $\mathbb{N}$)

Bud $A$ mnozina, $I$ mnozina indexov. Pre $i \in I$ bud $\omega_{i}$ $n_{i}$-arnou operaciou na A,
$n_{i} \in \mathbb{N}_{0}$. Potom $\mathcal{A} := (A, (\omega_{i})_{i \in \{1,..,n\}})$ oznacuje
\emph{algebru s nosnou mnozinou $A$ a suborom operacii $(\omega_{i})_{i \in I} =: \Omega$}.
Casto byva $I$ konecna, potom piseme
\begin{equation*}
	(A,\Omega) = (A, (\omega_{i})_{i \in I}) =: (A, \omega_{1},...,\omega_{n})
\end{equation*}

\textbf{Neutralny prvok} - bud $A$ mnozina, $\circ$ binarna operacia na $A$. Prvok $e \in A$ sa nazyva
neutralny prvok vzhladom k $\circ: \Leftrightarrow \forall x \in A: e \circ x = x \circ e = x$ (moze byt
aj lavy neutralny $e \circ x = x$, alebo aj pravy neutralny $x \circ e = x$)

\textbf{Zakon} - Rovnice, ktore maju tvar $t_{1}(x,y,z,...) = t_{2}(x,y,z,...)$ s vhodnymi termami
$t_{1}, t_{2}$ a musia byt splnene pre vsetky prvky uvazovanej algebry

\textbf{Inverzny prvok} - bud $A$ mnozina, $\circ$ binarna operacia na $A$, $e$ neutralny prvok, $x \in A$. Potom
nazyvame prvok $y \in A$ inverzny k prvku $x: \Leftrightarrow x \circ y = y \circ x = e$. (Moze byt aj lavy inverzny
$y \circ x = e$, alebo aj pravy inverzny $x \circ y = e$.)

\textbf{Asociativny zakon} - bud $A$ mnozina, $\circ$ binarna operacia na $A$, $\circ$ sa nazyva asociativna:
$\Leftrightarrow \forall x,y,z \in A: (x \circ y) \circ z = x \circ (y \circ z)$. Pokial je operacia asociativna,
existuje ku kazdemu prvku nanajvys 1 inverzny prvok.

\textbf{Operacia s delenim} - binarna operacia $\circ$ sa nazyva operacia s delenim: $\Leftrightarrow
\forall (a,b) \in A^{2} \exists (x,y) \in A^{2}: a \circ x = b \text{(lavy zakon o deleni)}
\land y \circ a = b \text{(pravy zakon o deleni)}.$

\textbf{Operacia s kratenim} - binarna operacia $\circ$ sa nazyva operacia s kratenim na $A$:
$\Leftrightarrow \forall a,x_{1},x_{2},y_{1},y_{2} \in A: (a \circ x_{1} = a \circ x_{2} \Rightarrow x_{1} = x_{2}) \text{(lavy zakon o krateni)}
\land (y_{1} \circ a = y_{2} \circ a \Rightarrow y_{1} = y_{2}) \text{(pravy zakon o krateni)}$

\textbf{Komutativny zakon} - binarna operacia $\circ$ sa nazyva komutativna:
$\Leftrightarrow \forall x,y \in A: x \circ y = y \circ x$

\textbf{Distributivny zakon} - pokial su $+, \cdot$ binarne operacie na $A$, potom sa $\cdot$ nazyva
distributivna nad $+: \Leftrightarrow \forall x,y,z \in A :
x \cdot (y + z) = x \cdot y + x \cdot z \land (y + z) \cdot x = y \cdot x + z \cdot x$

\subsection{Typy algebier}
\begin{itemize}
	\item \textbf{Grupoid} - algebra $(A, \cdot)$ typu (2)
	\item \textbf{Pologrupa} - grupoid $(H, \cdot)$ a $\cdot$ je asociativna operacia
	\item \textbf{Monoid} - pologrupa $(H, \cdot)$ a existuje neutralny prvok
	\item \textbf{Grupa} - monoid $(G, \cdot)$ pokial pre kazdy prvok z $G$ existuje prvok inverzny
	\item \textbf{Komutativna/abelovska grupa} - grupa $(G, \cdot)$ (resp. $(G, \cdot, e, \textsuperscript{-1}$)) kde pre $\cdot$ plati komutativny zakon
	\item \textbf{Okruh} - algebra $(R,+,\cdot)$ (resp. $(R,+,0,-,\cdot)$) typu (2,2) (resp. (2,0,1,2)) pokial
		\begin{itemize}
			\item $(R, +)$ (resp. $(R,+,0,-)$) je abelovska grupa
			\item $(R, \cdot)$ je pologrupa
			\item $\cdot$ je distributivny nad $+$
		\end{itemize}
	\item \textbf{Okruh s jednotkovym prvkom} - algebra $(R,+,0,-,\cdot,1)$ typu (2,0,1,2,0) pokial
		\begin{itemize}
			\item $(R,+,0,-,\cdot)$ je okruh
			\item 1 je neutralny prvok vzhladom k $\cdot$
		\end{itemize}
	\item \textbf{Komutativny okruh} - pre $\cdot$ plati komutativita
	\item \textbf{Komutativny okruh s jednotkovym prvkom} - vyssie 2 dokopy
	\item \textbf{Obor integrity} - komutativny okruh s jednotkovym prvkom $(R,+,0,-,\cdot,1)$ pokial
		\begin{itemize}
			\item $R \setminus \{0\} \not= \varnothing$ (teda $0 \not= 1$)
			\item $\forall x,y \in R: x \not= 0 \land y \not= 0 \Rightarrow xy \not= 0$ (neexistuju delitele nuly)
		\end{itemize}
	\item \textbf{Teleso} - okruh s jednotkovym prvkom $(R,+,0,-,\cdot,1)$ pokial
		\begin{itemize}
			\item $0 \not= 1$
			\item $(R \setminus \{0\}, \cdot)$ je grupa
		\end{itemize}
	\item \textbf{Pole} - komutativne teleso ($(R \setminus \{0\}, \cdot)$ musi byt abelovska grupa)
	\item \textbf{Svaz} - algebra $(V,\cap,\cup)$ typu (2,2), ak $\forall a,b,c \in V$ plati
		\begin{itemize}
			\item $a \cap b = b \cap a, a \cup b = b \cup a$
			\item $a \cap (b \cap c) = (a \cap b) \cap c, a \cup (b \cup c) = (a \cup b) \cup c$
			\item $a \cap (a \cup b) = a, a \cup (a \cap b) = a$
		\end{itemize}
	\item \textbf{Distributivny svaz} - musi platit
		\begin{equation*}
			a \cap (b \cup c) = (a \cap b) \cup (a \cap c), a \cup (b \cap c) = (a \cup b) \cap (a \cup c)
		\end{equation*}
	\item \textbf{Ohraniceny svaz}
		\begin{itemize}
			\item \textbf{Nulovy prvok svazu} - $\forall a \in A: a \cup 0 = a$
			\item \textbf{Jednotkovy prvok svazu} - $\forall a \in A: a \cap 1 = a$
		\end{itemize}
	\item \textbf{Komplementarny ohraniceny svaz} - $\forall a \in A: a \cap a' = 0 \land a \cup a' = 1$,
		$a'$ sa nazyva komplement prvku $a$. Nazyva sa aj \textbf{Booleovsky svaz}.
	\item \textbf{Booleova algebra} - ak je $(B,\cap,\cup,0,1)$ booleovsky svaz, tak $(B,\cap,\cup,0,1,')$ je booleova
		algebra
	\item \textbf{Vektorovy priestor} - bud $(K,+,0,-,\cdot,1)$ pole, $I = \{a,b,c\} \cup K$, kde $a,b,c \not\in K$,
		, $a,b,c$ po dvoch rozne. Algebra $(V,(\omega_{i})_{i \in I})$ typu (2,0,1,$\text{(1)}_{\lambda \in K}$) sa nazyva
		vektorovy priestor nad $K$, iff
		\begin{itemize}
			\item $(V,\omega_{a},\omega_{b},\omega_{c}) =: (V,+,0,-)$ je abelovska grupa
			\item \begin{flalign*}
					&\forall x,y \in V, \lambda,\mu \in K : & \\
					&\omega_{\lambda}(x + y) = \omega_{\lambda}(x) + \omega_{\lambda}(y) & \\
					&\omega_{\lambda + \mu}(x) = \omega_{\lambda}(x) + \omega_{\mu}(y) & \\
					&\omega_{\lambda\mu}(x) = \omega_{\lambda}(\omega_{\mu}(x)) & \\
					&\omega_{1}(x) = x &
				\end{flalign*}
		\end{itemize}
\end{itemize}
\subsection{Zakladne pojmy teorie grup}
\textbf{Sucin} - Bud $(G,\cdot)$ grupoid, $a_{1},...,a_{n} \in G (n \in \mathbb{N})$. \emph{Sucin} $a_{1},...,a_{n}$
je definovany indukciou vztahom $a_{1}...a_{n} := (a_{1}...a_{n-1})a_{n}$

\textbf{Mocnina} - Bud $(G,\cdot)$ grupoid, mocnina prvku $a$ je definovana ako $a^{1} := a, a^{n+1} := (a^{n})a (n \in \mathbb{N})$

Bud $(G,\cdot,e,\textsuperscript{-1})$ grupa, $a,b \in G$, potom plati $(ab)^{-1} = b^{-1}a^{-1}$

Bud $(G,\cdot,e,\textsuperscript{-1})$ grupa, $a \in G$, potom plati $a^{0} = e$ a $a^{-n} = (a^{-1})^{n}$

Pravidla pre pocitanie s mocninami v grupach
\begin{align*}
	a^{n}a^{m} &= a^{n + m} \\
	(a^{n})^{m} &= a^{nm} \\
	(ab)^{n} &= a^{n}b^{n} \text{pokial je } \cdot \text{ komutativna}
\end{align*}

\textbf{Rad prvku} - bud $(G,\cdot,e,\textsuperscript{-1})$ grupa, $a \in G$, potom kardinalne cislo
	$o(a) = \vert\{a^{0} = e,a^{1},a^{-1},a^{2},a^{-2},...\}\vert = \vert \{a^{k} \pipesep k \in \mathbb{Z}\}\vert$
	sa nazyva rad prvku $a$

\textbf{Delenie so zvyskom} - $\forall k,l \in \mathbb{Z}, l \not= 0, \exists q,r \in \mathbb{Z}:
0 \le r < \vert l \vert \land k = lq + r$

\textbf{Symetricka grupa} - Bud $M$ mnozina a $S_{M} = \{ f : M \to M \pipesep f \text{bijektivne} \}$.
$(S_{M}, \circ, id_{M}, \textsuperscript{-1})$ je grupa, ktora sa nazyva symetricka grupa na $M$.
Prvky mnoziny $S_{M}$ sa nazyvaju permutacie mnoziny $M$. Ak $M = \{1,2,...,n\}$ tak piseme $S_{n}$. Napr
\begin{equation*}
	S_{3} = \begin{Bmatrix}
	\begin{pmatrix}
		1 & 2 & 3 \\
		1 & 2 & 3
	\end{pmatrix},
	\begin{pmatrix}
		1 & 2 & 3 \\
		2 & 3 & 1
	\end{pmatrix},
	\begin{pmatrix}
		1 & 2 & 3 \\
		3 & 1 & 2
	\end{pmatrix},
	\begin{pmatrix}
		1 & 2 & 3 \\
		1 & 3 & 2
	\end{pmatrix},
	\begin{pmatrix}
		1 & 2 & 3 \\
		3 & 2 & 1
	\end{pmatrix},
	\begin{pmatrix}
		1 & 2 & 3 \\
		2 & 1 & 3
	\end{pmatrix}
	\end{Bmatrix}
\end{equation*}
Cyklicky zapis:
\begin{equation*}
	S_{3} = \{ (1), (123), (312), (23), (13), (12) \}
\end{equation*}

\end{document}
