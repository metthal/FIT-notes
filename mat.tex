\documentclass[12pt]{article}

\usepackage[margin=1in]{geometry} 
\usepackage{amsmath,amsthm,amssymb,amsfonts}
\usepackage[utf8]{inputenc}
\usepackage[slovak]{babel}

\newcommand{\pipesep}{\hspace{3pt} \vert \hspace{3pt}}
\setlength\parindent{0pt}

\begin{document}

\title{Poznamky z MATu}
\author{Marek Milkovic}
\date{}
\maketitle

\section{Vyrokova logika}

Vyrokova logika skuma sposob tvorby zlozenych vyrokov z danych jednoduchych vyrokov a zavislosti
pravdivosti zlozeneho vyroku na pravdivosti vyrokov z ktorych je zlozeny.
\\
$P$ - neprazdna mnozina symbolov - \emph{prvotnych vyrokov} \\
Prvotne vyroky predstavuju jednoduche vyroky. Zlozene vyroky sa skladaju z jednoduchych pomocou
logickych spojek - $\neg, \land, \lor, \to, \leftrightarrow$. \\
$L_{P}$ - jazyk vyrokovej logiky - prvky $P$, logicke spojky a zatvorky $()$. \\
Ulohu zlozenych vyrokov hraju \emph{vyrokove formule} jazyka $L_{P}$, ktore vznikaju ako:
\begin{enumerate}
		\item Kazda prvotna formula $p \in P$ je vyrokova formula
		\item Ak su $A,B$ vyrokove formule, tak aj $\neg A$, $A \land B$, $A \lor B$,
			$A \to B$, $A \leftrightarrow B$.
		\item Kazda vyrokova formula vznikne konecnym poctom pouzitia pravidiel 1 a 2
\end{enumerate}
\vspace{1cm}
Pravdivostne ohodnotenie prvotnych formuli - $v: P \to \{0,1\}$ \\
Rozsirenie na vsetky formule - $\bar{v}$
\begin{enumerate}
		\item $\bar{v}(p) = v(p) \forall p \in P$
		\item Ak su $A,B$ vyrokove formule, tak $\bar{v}(\neg A), \bar{v}(A \land B),
			\bar{v}(A \lor B), \bar{v}(A \to B), \bar{v}(A \leftrightarrow B)$ sa definuje pomocou
			pravdivostnej tabulky v zavislosti na hodnote $\bar{v}(A), \bar{v}(B)$
			(a tie pozname velmi dobre)
\end{enumerate}
Vyrokova formula $A$ je pravdiva pri ohodnoteni $v$ - $\bar{v}(A) = 1$ \\
Vyrokova formula $A$ je tautologia ak $\bar{v}(A) = 1$ pre lubovolne ohodnotenie $v$ - $\models A$\\
Vyrokove formule su \emph{ekvivalentne} prave vtedy ak $\bar{v}(A) = \bar{v}(A)$ pre lubovolne
	ohodnotenie $v$ - $A \leftrightarrow B$ musi byt tautologia \\
Kazda vyrokova formula je logicky ekvivalentna niektorej vyrokovej formuli, v ktorej su len spojky
	$\neg, \to$ (ale aj $\neg, \land$ a $\neg, \lor$) \\
\begin{enumerate}
\item Nicodova spojka - $\downarrow$ - NOR
\item Shefferova spojka - $\vert$ - NAND
\end{enumerate}
\subsection{Dokazatelnost vo vyrokovej logike}
Vyrokova logika bude budovana ako formalna axiomaticka teoria \\
\textbf{Abeceda}
\begin{enumerate}
	\item mnozina $P$
	\item logicke spojky $\neg, \to$
	\item pomocne symboly $()$
\end{enumerate}
\textbf{Formule}
\begin{enumerate}
	\item vsetky prvotne formule su formule
	\item ak su $A$, $B$ formule tak aj $\neg A$ a $A \to B$ su formule
	\item opakovanim 1 a 2 vznikaju formule
\end{enumerate}
\textbf{Jazyk} - abeceda a formule tvoria jazyk \\
\textbf{Axiomy} - nekonecne mnoho axiomu zadanych pomocou 3 schemat
\begin{itemize}
	\item (A1) $A \to (B \to A)$
	\item (A2) $(A \to (B \to C)) \to ((A \to B) \to (A \to C))$
	\item (A3) $(\neg B \to \neg A) \to (A \to B)$
\end{itemize}
\textbf{Odvodzovacie pravidlo} - \emph{Modus ponens} - pravidlo odlucenia -
	z formuli $A, (A \to B)$ sa odvodi $B$. $A, (A \to B)$ su predpoklady, $B$ je zaver \\
\textbf{Dokaz} - lubovolna konecna postupnost $A_{1}, ..., A_{n}$ vyrokovych formuli takych,
	ze $\forall i \le n$ je $A_{i}$ bud axiom, alebo zaver pravidla modus ponens \\
- Formula $A$ je \emph{dokazatelna}, ak existuje dokaz, ktoreho poslednou formulou je $A$
- $\vdash A$ \\
- Kazda dokazatelna formula vo vyrokovej logike je tautologia \\
\vspace{0.5cm}
Nech $T$ je mnozina formuli vyrokovej logiky. Hovorime, ze konecna postupnost $A_{1},...,A_{n}$
je \emph{dokazom formule A z predpokladu T}, ak $A_{n}$ je formula $A$ a pre lubovolne
$i \le n$ plati ze $A_{i}$ je bud axiom, alebo formula z $T$ alebo $A_{i}$ je zaverom
modus ponens - formula $A$ je \emph{dokazatelna z predpoklatu T} - $T \vdash A$ \\
\textbf{Veta o dedukcii} - Nech $T$ je mnozina formuli, nech $A,B$ su formule, potom
	$T \vdash A \to B$ prave vtedy ked $T \cup \{A\} \vdash B$. \\
\textbf{Veta o uplnosti} - pre lubovolnu formulu $A$ vo vyrokovej logike plati $\vdash A$,
prave ked $\models A$.

\section{Predikatova logika 1.radu}
\textbf{Premenne} - prvky z nejakej mnoziny \\
\textbf{Funkcne symboly} - $f,g,h...$ - operacie, $n$-arny symbol \\
\textbf{Predikat} - vztah medzi urcitym poctom objektov \\
\textbf{Predikatovy symbol} - $p,q,r...$ - vyjadrujeme nimi predikaty - $n$-arny symbol \\
\textbf{Atomicka formula} - zlozena premennych, konstant, funkcnych symbolov
	a predikatovych symbolov\\
\textbf{Logicke spojky} - rovnake ako vo vyrokovej logike\\
\textbf{Kvantifikatory premennych} - univerzalny $\forall$ a existencny $\exists$\\
\textbf{Abeceda jazyka predikatovej logiky} - premenne, konstanty, funkcne a predikatove symboly,
	logicke spojky, kvantifikatory a pomocne symboly $()$\\
Premenna predikatovej logiky 1. radu predstavuje konkretny objekt, nedokaze predstavovat
inu mnozinu \\

\textbf{Term}
\begin{enumerate}
	\item Kazda premenna je term
	\item Ak je $f$ $n$-arny funkcny symbol, a $t_{1},...,t_{n}$ su termy, tak aj
		$f(t_{1},...,t_{n})$ je term
	\item Konecny uzitim pravidiel 1 a 2 vznikne term
\end{enumerate}

\textbf{Atomicka formula}\\
Ak je $p$ $n$-arny predikatovy symbol a $t_{1},...,t_{n}$ su termy, tak $p(t_{1},...,t_{n})$
je \emph{atomicka formula} \\

\textbf{Formule} \\
\begin{enumerate}
	\item Kazda atomicka formula je formula
	\item Ak su $\varphi, \psi$ formule, tak aj $\neg \varphi, \varphi \land \psi, \varphi \lor \psi,
		\varphi \to \psi, \varphi \leftrightarrow \psi$ su formule
	\item Ak je $x$ premenna a $\varphi$ formula, tak $\forall x \varphi$ a $\exists x \varphi$
		su formule
	\item Kazda formula vznikne konecnym uzitim formuli 1-3
\end{enumerate}
Vyskyt premennej $x$ vo formuli $\varphi$ sa nazyva viazany, ak sa nachadza v nejakej podformuli
v tvare $\forall x \psi$ alebo $\exists x \psi$. V opacnom pripade sa jedna o vyskyt volny.
Formula co neobsahuje ziadnu volnu premennu sa nazyva \emph{uzavrena formula} alebo aj \emph{vyrok}.
\subsection{Semantika predikatovej logiky}
Chceme dat intepretaciu symbolom jazyka predikatovej logiky 1. radu. Vymedzime teda obor,
ktory bude urcovat mozne hodnoty premennych $M$. Funkcnym symbolom budu odpovedat operacie na $M$.
Predikatovym symbolom budu odpovedat vztahy medzi objektami $M$, ktore je mozne popisat ako
relacie na $M$. \\
\\
\textbf{Realizacia jazyka} - Nech je $L$ jazyk 1. radu. Realizaciou jazyka rozumieme algebraicku
strukturu $\mathcal{M}$, ktora sa sklada z
\begin{enumerate}
	\item neprazdnej mnoziny $M$, ktoru nazveme \emph{univerzum}
	\item pre kazdy funkcny symbol $f$ pocetnosti $n$ je dane zobrazenie
		$f_{\mathcal{M}}: M^{n} \to M$
	\item pre kazdy predikatovy symbol $p$ pocetnosti $n$, okrem rovnosti, je dana relacia
		$p_{\mathcal{M}} \subseteq M^{n}$
\end{enumerate}
\textbf{Ohodnotenie premennych} - Lubovolne zobrazenie $e$ mnoziny vsetkych premennych do
univerza $M$ dane realizaciou $\mathcal{M}$ jazyka $L$ \\
- Ak je $x$ premenna, $e$ ohodnotenie premennych, $m \in M$, potom znacime $e(x/m)$ \\
- \textbf{Hodnotu termu} $t$ v realizacii $\mathcal{M}$ pri danom ohodnoteni $e$ znacime
	$t[e]$ a definovana je nasledovne
	\begin{enumerate}
		\item Ak je $t$ premenna $x$ potom $t[e]$ je $e(x)$
		\item Ak je $t$ v tvare $f(t_{1},...,t_{n})$, kde $f$ je $n$-arny funkcny symbol
			a $t_{i}$ su termy, potom $t[e]$ je $f_{\mathcal{M}}(t_{1}[e],...,t_{n}[e])$
	\end{enumerate}
\vspace{0.5cm}
Nech $\mathcal{M}$ je realizacia jazyka $L$, nech $e$ je ohodnotenie premennych a $\varphi$
je formula jazyka $L$. Definujeme, co znamena \emph{formula $\varphi$ je pravidva v
$\mathcal{M}$ pri ohodnoteni e} - $\mathcal{M} \models \varphi[e]$
\begin{enumerate}
	\item Ak je $\varphi$ atomicka formula v tvare $p(t_{1},...,t_{n})$, tak
		$\mathcal{M} \models \varphi[e]$ prave ked $(t_{1}[e],...,t_{n}[e]) \in p_{\mathcal{M}}$
	\item Ak je $\varphi$ atomicka formula v tvare $t_{1} = t_{2}$, tak
		$\mathcal{M} \models \varphi[e]$ prave ked $t_{1}[e]$ je ten isty prvok ako $t_{2}[e]$
	\item Ak je $\varphi$ v tvare $\neg \psi$, tak 
		$\mathcal{M} \models \varphi[e]$ prave ked $\mathcal{M} \not\models \psi[e]$
	\item Ak je $\varphi$ v tvare $\eta \land \psi$, $\eta \lor \psi$, $\eta \to \psi$,
		$\eta \leftrightarrow \psi$ tak $\mathcal{M} \models (\eta \land \psi)[e]$ prave vtedy ak
		$\mathcal{M} \models \eta[e]$ a zaroven $\mathcal{M} \models \psi[e]$ a analogicky
		pre zvysok
	\item Ak je $\varphi$ v tvare $\forall x \psi$, tak
		$\mathcal{M} \models \varphi[e]$ prave vtedy ak
		$\forall m \in M : \mathcal{M} \models \psi[e(x/m)]$
	\item Ak je $\varphi$ v tvare $\exists x \psi$, tak
		$\mathcal{M} \models \varphi[e]$ prave vtedy ak
		$\exists m \in M : \mathcal{M} \models \psi[e(x/m)]$
\end{enumerate}
Formula $\varphi$ je \emph{splnena} v realizacii $\mathcal{M}$, ak je $\varphi$ pravdiva
v $\mathcal{M}$ pri kazdom ohodnoteni $e$. Potom piseme $\mathcal{M} \models \varphi$.

Ak je $\varphi$ uzavrena formula, tak vravime ze $\varphi$ je \emph{pravdiva} v $\mathcal{M}$.

Formula sa nazyva \emph{splnitelna}, ak je splnena v nejakej realizacii.

Formula je \emph{logicky platna}, ak je splnena v kazdej realizacii jazyka $L$ - $\models \varphi$

Formula $\varphi, \psi$ su \emph{logicky ekvivalentne} ak v lubovolnej realizacii $\mathcal{M}$
a pri lubovolnom ohodnoteni $e$ je $\mathcal{M} \models \varphi[e]$ prave vtedy ak
$\mathcal{M} \models \psi[e]$.

\subsection{Substitucia termov za premenne}
Ak je $\varphi$ formula, $x$ je premenna a $t$ je term, potom vyraz ktory vznikne z formule
$\varphi$ nahradenim kazdeho volneho vyskytu premennej $x$ termom $t$ je opat formula.

Term $t$ je \emph{substituovatelny} za $x$ do formule $\varphi$, ak ziadny volny vyskyt premennej
$x$ sa nenachadza v obore kvatifikacie nejakeho kvantifikatoru premennej $y$, kde $y$ je
premenna obsiahnuta v terme $t$.

Budeme znacit $\varphi_{x}[t]$ formulu, ktora vznikne z $\varphi$ nahradenim kazdeho volneho
vyskytu $x$ termom $t$.

Ak mame $\varphi, x, t$ a $t$ je substituovatelny za $x$ do $\varphi$, potom
$(\forall x \varphi) \to \varphi_{x}[t]; \varphi_{x}[t] \to (\exists x \varphi)$ su logicky
platne formule.

\subsection{Formalny system predikatovej logiky}
Zavedieme predikatovu logiku ako formalny axiomaticky system. Obmedzime sa len na logicke spojky
$\neg, \to$ a kvantifikatoru $\forall$.

\textbf{Schema vyrokovych axiomov}
\begin{itemize}
	\item $\varphi \to (\psi \to \varphi)$
	\item $(\varphi \to (\psi \to \eta)) \to ((\varphi \to \psi) \to (\varphi \to \eta))$
	\item $(\neg\psi \to \neg\varphi) \to (\varphi \to \psi)$
\end{itemize}
\textbf{Schema axiomu kvantifikatoru}
\begin{itemize}
	\item $(\forall x(\varphi \to \psi)) \to (\varphi \to (\forall x \psi))$
		($x$ nema volny vyskyt v $\varphi$)
\end{itemize}
\textbf{Schema axiomu substitucie}
$t$ je substituovatelny za $x$ do $\varphi$
\begin{itemize}
	\item $(\forall x \varphi) \to \varphi_{x}[t]$
\end{itemize}
\textbf{Schema axiomov rovnosti}
\begin{itemize}
	\item $x = x$
	\item $(x_{1} = y_{1} \to (x_{2} = y_{2} \to ( ... (x_{n} = y_{n} \to
		f(x_{1}, x_{2}, ..., x_{n}) = f(y_{1}, y_{2}, ..., y_{n}))...)))$
	\item $(x_{1} = y_{1} \to (x_{2} = y_{2} \to ( ... (x_{n} = y_{n} \to
		p(x_{1}, x_{2}, ..., x_{n}) \to p(y_{1}, y_{2}, ..., y_{n}))...)))$
\end{itemize}
\subsection{Odvodzovacie pravidla predikatovej logiky}
\begin{itemize}
	\item \textbf{Pravidlo odlucenia - modus ponens} - Z formuli $\varphi$,$\varphi \to \psi$
		sa odvodi formula $\psi$
	\item \textbf{Pravidlo zobecnenia - generalizacia} - Pre lubovolnu premennu $x$ sa z formule
		$\varphi$ odvodi formula $(\forall x \varphi)$
\end{itemize}

\textbf{Dokaz} - lubovolna postupnost $\varphi_{1},...,\varphi_{n}$ formuli jazyka $L$ v ktorej
pre kazdy index $i$ je formula $\varphi_{i}$ bud axiom predikatovej logiky alebo je ju mozne
odvodit z niektorych predchadzajucich formuli pouzitim pravidla odlucenia alebo zobecnenia.

Formula $\varphi$ je \emph{dokazatelna}, pokial existuje dokaz, ktoreho poslednou formulou je
$\varphi$. Piseme $\vdash \varphi$.

Bud $T$ mnozina formuli jazyka $L$. Formula $\varphi$ je \emph{dokazatelna z predpokladov T}, ak
existuje dokaz z predpokladu $T$, teda postupnost formuli, ze jej poslednou formulou je $\varphi$
a kazda predosla formula je bud axiom, alebo je to formula z $T$ alebo vznikla uzitim pravidla
odlucenia alebo zobecnenia. Piseme $T \vdash \varphi$.

Lubovolna formula jazyka $L$ dokazatelna v predikatovej logike 1. radu je logicky platnou tj.
splnena v kazdej realizacii jazyka $L$.

\textbf{Lemma} - Ak $\vdash \varphi \to \psi$ a $x$ nema volny vyskyt v $\varphi$, tak
$\vdash \varphi \to (\forall x \psi)$

\textbf{Lemma} - Ak $\vdash \varphi \to \psi$ a $x$ nema volny vyskyt v $\varphi$, tak
$\vdash (\exists x \varphi) \to \psi$

\textbf{Lemma} - Ak $\varphi$ je formula, $x$ je premenna a $t$ term substituovatelny za $x$
do $\varphi$, tak $\vdash \varphi_{x}[t] \to (\exists x \varphi)$
\end{document}
