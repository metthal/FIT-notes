\documentclass[12pt]{article}

\usepackage[margin=1in]{geometry} 
\usepackage{amsmath,amsthm,amssymb,amsfonts}
\usepackage[utf8]{inputenc}
\usepackage[slovak]{babel}
\usepackage{tikz}

\newcommand{\pipesep}{\hspace{3pt} \vert \hspace{3pt}}
\setlength\parindent{0pt}

\begin{document}

\title{Poznamky z TINu}
\author{Marek Milkovic}
\date{}
\maketitle

\section{Formalne jazyky}
Nebudem tu vysvetlovat co je abeceda, retazec, konkatenacia atd. To su trivialne pojmy, ktore musi kazdy vediet.

Algebra jazykov $<2^{\Sigma^{*}}, \cup, \cdot, \varnothing, \{\varepsilon\}>$ tvori polookruh (ako okruh, len nemusi sa splnit
podmienka inverznych prvkov)
\begin{itemize}
	\item $<2^{\Sigma^{*}}, \cup, \varnothing>$ je komutativny monoid
	\item $<2^{\Sigma^{*}}, \cdot, \{\varepsilon\}>$ je monoid
\end{itemize}

\subsection*{Gramatiky}
Gramatika je stvorica $(N, \Sigma, P, S)$
\begin{itemize}
	\item $N$ - konecna mnozina neterminalnych symbolov
	\item $\Sigma$ - konecna mnozina terminalnych symbolov
	\item $P$ - konecna mnozina prepisovancich pravidiel v tvare $(N \cup \Sigma)^{*}N(N \cup \Sigma)^{*} \times (N \cup \Sigma)^{*}$
	\item $S$ - startovaci neterminal, $S \in N$
\end{itemize}

Prepisovacie pravidla $(\alpha, \beta) \in P$ zapisujeme $\alpha \to \beta$

\textbf{Priama derivacia} - nech $G = (N, \Sigma, P, S)$ a nech $\lambda, \nu \in (N \cup \Sigma)^{*}$.
$\lambda,\nu$ mozeme zapisat v tvare $\lambda = \gamma\alpha\delta, \nu = \gamma\beta\delta$. Ak $\alpha \to \beta \in P$,
tak hovorime ze $\lambda$ priamo derivuje $\nu$ a zapisujeme $\lambda \Rightarrow \nu$.

\textbf{Vetna forma} - nech $G = (N, \Sigma, P, S)$ a $\alpha \in (N \cup \Sigma)^{*}$. Ak $S \Rightarrow^{*} \alpha$,
tak $\alpha$ sa nazyva vetna forma. Vetou sa rozumie $\alpha \in \Sigma^{*}$.

\subsection{Chomskeho hierarchia}
\begin{itemize}
	\item \textbf{Gramatiky typu 0}
		\begin{itemize}
			\item $\alpha \to \beta$
			\item $\alpha \in (N \cup \Sigma)^{*}N(N \cup \Sigma)^{*}$
			\item $\beta \in (N \cup \Sigma)^{*}$
		\end{itemize}
	\item \textbf{Gramatiky typu 1}
		\begin{itemize}
			\item $\alpha A\beta \to \alpha\gamma\beta$
			\item $\alpha,\beta \in (N \cup \Sigma)^{*}$
			\item $\gamma \in (N \cup \Sigma)^{+}$
			\item $A \in N$
		\end{itemize}
	\item \textbf{Gramatiky typu 2}
		\begin{itemize}
			\item $A \to \alpha$
			\item $A \in N$
			\item $\alpha \in (N \cup \Sigma)^{*}$
		\end{itemize}
	\item \textbf{Gramatiky typu 3}
		\begin{itemize}
			\item $A \to xB$ alebo $A \to x$
			\item $A,B \in N$
			\item $x \in \Sigma^{*}$
		\end{itemize}
\end{itemize}

\section{Regularne jazyky}
\subsection*{Konecny automat}
Patica $(Q,\Sigma,\delta,q_{0},F)$
\begin{itemize}
	\item $Q$ - konecna mnozina stavov
	\item $\Sigma$ - konecna vstupna abeceda
	\item $\delta$ - prechodova funkcia $Q \times \Sigma \to 2^{Q}$
	\item $q_{0}$ - pociatocny stav $q_{0} \in Q$
	\item $F$ - mnozina koncovych stavov $F \subseteq Q$
\end{itemize}

Deterministicky KA v pripade ze $\forall q \in Q \forall a \in \Sigma: |\delta(q,a)| \le 1$. Potom
mozeme DKA definovat tak, ze prechodova funkcia bude zobrazenie $Q \times \Sigma \to Q$. AK je $\delta$
definovana pre kazde $q$, tak automat nazyvame uplne definovany DKA.

\textbf{Konfiguracia automatu} - majme KA $M = (Q,\Sigma,\delta,q_{0},F)$. Konfiguraciou automatu
nazyvame dvojicu $(q, w) \in Q \times \Sigma^{*}$.

\textbf{Prechod automatu} - relacia prechodu $\vdash \subseteq (Q \times \Sigma^{*}) \times (Q \times \Sigma^{*})$,
ktora je definovana nasledovne
\begin{equation*}
	\forall q_{1}, q_{2} \in Q \forall w_{1},w_{2} \in \Sigma^{*}:
	(q_{1}, w_{1}) \vdash (q_{2}, w_{2}) \overset{def}{\Leftrightarrow} \exists a \in \Sigma: w_{1} = aw_{2} \land q_{2} \in \delta(q_{1}, a)
\end{equation*}

\textbf{Jazyk prijmany automatom} - $L(M) = \{w \in \Sigma^{*} \pipesep (q_{0}, w) \vdash^{*} (q_{F}, \varepsilon) \land q_{F} \in F\}$

\subsection*{Prevod NKA na ekvivalentny DKA}
\begin{itemize}
	\item Vstup NKA $M = (Q,\Sigma,\delta,q_{0},F)$
	\item Vystup DKA $M' = (Q',\Sigma,\delta',q'_{0},F')$
	\begin{itemize}
		\item $Q' = 2^{Q} \setminus \{\varnothing\}$
		\item $\forall T \in 2^{Q} \setminus \{\varnothing\} \forall a \in \Sigma: \delta'(T,a) = \underset{q \in T}{\bigcup}\delta(q,a)$ ale pre $\varnothing$ nedefinovana
		\item $q'_{0} = \{q_{0}\}$
		\item $F' = \{T \in 2^{Q} \setminus \{\varnothing\} \pipesep T \cap F \not= \varnothing\}$
	\end{itemize}
\end{itemize}

\subsection*{Konecne automaty a jazyky typu 3}
Gramatika $G = (N,\Sigma,P,S)$ s pravidlami v tvare $A \to xB|x$ resp. $A \to Bx|x$, kde $A,B \in N$
a $x \in \Sigma^{*}$ sa nazyva \textbf{prava linearna} resp. \textbf{lava linearna}.

Gramatika $G = (N,\Sigma,P,S)$ s pravidlami v tvare $A \to aB|a$ resp. $A \to Ba|a$, kde $A,B \in N$
a $a \in \Sigma$ sa nazyva \textbf{prava regularna} resp. \textbf{lava regularna}.

Gramatika $G = (N,\Sigma,P,S)$ s pravidlami v tvare $A \to xBy$ kde $x,y \in \Sigma^{*}$ a $A,B \in N$
sa nazyva \textbf{linearna gramatika}.

Kazda gramatika $G = (N,\Sigma,P,S)$ moze byt prevedna na gramatiku taku, ze jej pravidla su v tvare
$A \to aB$ a $A \to \varepsilon$, kde $A,B \in N$ a $a \in \Sigma$.

\emph{Dokaz} Zostavme gramatiku $G' = (N',\Sigma,P',S')$
\begin{itemize}
	\item Pravidla v tvare $A \to aB$ a $A \to \varepsilon$ len prekopirujeme
	\item Pre pravidla v tvare $A \to a_{1}a_{2}...a_{n}B, n \ge 2$ vytvorime pravidla
		\begin{itemize}
			\item $A \to a_{1}A_{1}$
			\item $A_{1} \to a_{2}A_{2}$
			\item ...
			\item $A_{n-1} \to a_{n}B$
		\end{itemize}
	\item Pre pravidla v tvare $A \to a_{1}a_{2}...a_{n}, n \ge 1$ vytvorime pravidla
		\begin{itemize}
			\item $A \to a_{1}A_{1}$
			\item $A_{1} \to a_{2}A_{2}$
			\item ...
			\item $A_{n-1} \to a_{n}A_{n}$
			\item $A_{n} \to \varepsilon$
		\end{itemize}
	\item Eliminujeme jednoduche pravidla $A \to B$
\end{itemize}

\subsection*{Prevod gramatiky typu 3 na NKA}
Nech $\mathcal{L}_{M}$ je mnozina vsetkych jazykov prijmanych konecnymi automatmi a nech $L$ je lubovolny
jazyk typu 3 ($L \in \mathcal{L}_{3}$). Potom existuje konecny automat $M$ taky, ze
\begin{equation*}
	L = L(M),\text{ tj. }\mathcal{L}_{3} \subseteq \mathcal{L}_{M}
\end{equation*}
\emph{Dokaz}
\begin{itemize}
	\item Mozeme predpokladat ze $L = L(G)$ pre gramatiku $G$, ktora ma pravidla len v tvare $A \to aB|\varepsilon$.
	\item Zostrojime automat $M = (Q,\Sigma,\delta,q_{0},F)$
		\begin{itemize}
			\item $Q = N$
			\item $\forall A,B \in N \forall a \in \Sigma: \delta(A,a) \in B \Leftrightarrow A \to aB \in P$
			\item $q_{0} = S$
			\item $F = \{ A \in N \pipesep A \to \varepsilon \in P\}$
		\end{itemize}
	\item Matematickou indukciou dokazeme, ze $L(G) = L(M)$
		\begin{equation*}
			\forall A \in N: A \Rightarrow^{i+1} w \Leftrightarrow (A, w) \vdash^{i} (C, \varepsilon), C \in F, w \in \Sigma^{*}
		\end{equation*}
	Pre $i = 0$ plati
		\begin{equation*}
			A \Rightarrow \varepsilon \Leftrightarrow (A, \varepsilon) \vdash^{0} (A, \varepsilon), A \in F
		\end{equation*}
	co urcite plati. Teraz predpokladajme, ze mame retazec dlzky $i > 0$, ktory nazveme $w$ a majme lubovolny symbol
	$a \in \Sigma$. Zapiseme indukcny predpoklad
		\begin{equation*}
			B \Rightarrow^{i+1} w \Leftrightarrow (B, w) \vdash^{i} (C, \varepsilon), C \in F
		\end{equation*}
	Nech je teraz prijmany retazec $aw$. Zapiseme
		\begin{equation*}
			A \Rightarrow aB \Rightarrow^{i+1} aw \Leftrightarrow (A,aw) \vdash (B, w) \vdash^{i} (C, \varepsilon), C \in F
		\end{equation*}
	Z definicie priamej derivacie a prechodu automatu ale vyplyva, ze mozme zapisat
		\begin{equation*}
			A \Rightarrow^{i+2} aw \Leftrightarrow (A,aw) \vdash^{i+1} (C, \varepsilon), C \in F
		\end{equation*}
	Tvrdenie teda plati aj pre $i + 1$. $L(G) = L(M)$.
\end{itemize}

\subsection*{Prevod NKA na gramatiku typu 3}
Nech $L = L(M)$ pre nejaky konecny automat M. Potom existuje gramatika $G$ typu 3 taku, ze
\begin{equation*}
	L = L(G)\text{ tj. }\mathcal{L}_{M} \subseteq \mathcal{L}_{3}
\end{equation*}
\emph{Dokaz}
\begin{itemize}
	\item Nech $M = (Q, \Sigma, \delta, q_{0}, F)$. Predpokladajme, ze $M$ je NKA. Nech $G = (Q,\Sigma,P,q_{0})$
		je gramatika, ktorej pravidla su definovane ako
	\item Ak $\delta(A,a) \in B$, tak $A \to aB \in P$
	\item Ak $A \in F$, tak $A \to \varepsilon \in P$
\end{itemize}

\subsection*{Eliminacia nedosiahnutelnych stavov}
\textbf{Nedosiahnutelny stav} - Nech $M = (Q, \Sigma, \delta, q_{0}, F)$ je KA, stav $q \in Q$ nazveme
dosiahnutelny, pokial $\exists w \in \Sigma^{*}: (q_{0}, w) \underset{M}{\vdash^{*}} (q, \varepsilon)$.
Stav nazveme nedosiahnutelny, pokial nie je dosiahnutelny.

\textbf{Algoritmus}
\begin{flalign*}
	&i := 1 & \\
	&S_{1} = \{q_{0}\} & \\
	&\texttt{repeat} & \\
	&\hspace{1cm} S_{i+1} = S_{i} \cup \{p \pipesep \exists q \in S_{i} \exists a \in \Sigma : p = \delta(q,a)\} & \\
	&\hspace{1cm} i := i + 1 & \\
	&\texttt{until }S_{i-1} = S_{i} & \\
	&M' = (S_{i}, \Sigma, \delta|S_{i}, q_{0}, S_{i} \cap F) &
\end{flalign*}

\subsection*{Jazykovo nerozlisitelne stavy}
\textbf{Rozlisitelnost stavov} - Nech $M = (Q, \Sigma, \delta, q_{0}, F)$ je UDKA. Hovorime ze retazec $w \in \Sigma^{*}$ rozlisuje
stavy $q_{1}, q_{2}$ pokial $(q_{1}, w) \vdash^{*} (q_{3}, \varepsilon) \land (q_{2}, w) \vdash^{*} (q_{4}, \varepsilon) \land
((q_{3} \in F \land q_{4} \not\in F) \lor (q_{3} \not\in F \land q_{4} \in F))$.

\textbf{k-nerozlisitelnost} - Hovorime ze stavy $q_{1}, q_{2} \in Q$ su k-nerozlisitelne a piseme
$q_{1} \overset{k}{\equiv} q_{2}$, prave ked neexistuje $w \in \Sigma^{*}, |w| \le k$, ktory rozlisuje $q_{1}$ a $q_{2}$.

\textbf{Formalna definicia k-nerozlisitelnosti} - Nech $M = (Q, \Sigma, \delta, q_{0}, F)$ je UDKA,
$\forall q_{1},q_{2} \in Q: q_{1} \overset{k}{\equiv} q_{2} \overset{def}{\Leftrightarrow}
(\forall w \in \Sigma^{*}: |w| \le k ((\exists q_{3}\in F: (q_{1},w) \vdash^{*} (q_{3}, \varepsilon)) \Leftrightarrow
(\exists q_{4} \in F: (q_{2},w) \vdash^{*} (q_{4}, \varepsilon))))$

\textbf{Nerozlisitelnost} - Stavy $q_{1}, q_{2}$ su nerozlisitelne, znacime $q_{1} \equiv q_{2}$, pokial su
k-nerozlisitelne pre $k \ge 0$

\textbf{Redukovany DKA} - UDKA nazyva redukovany, pokial neobsahuje nedosiahnutelne stavy a ziadne
dva stavy su nerozlisitelne
\emph{Dokaz} pre $|Q| = n, n \le 2$ a plati $\forall q_{1},q_{2}: q_{1} \equiv q_{2} \Leftrightarrow q_{1} \overset{k}{\equiv} q_{2}$.
Dokaz pre $"\Rightarrow"$ je trivialny a vyplyva z definicie vseobecnej $\equiv$. Dokaz pre $"\Leftarrow"$.
\begin{enumerate}
	\item Ak $|F| = 0$ alebo $|F| = n$ tak ziadne 2 stavy nemozu byt rozlisitelne. Takze plati.
	\item Ak $0 < |F| < n$ tak ukazeme ze plati $\equiv = \overset{n-2}{\equiv} \subseteq \overset{n-1}{\equiv}
		\subseteq ... \subseteq \overset{1}{\equiv} \subseteq \overset{0}{\equiv}$:
		\begin{itemize}
			\item Plati ze $\forall q_{1}, q_{2} \in Q: q_{1} \overset{0}{\equiv} q_{2} \Leftrightarrow
				(q_{1} \in F \land q_{2} \not\in F) \lor (q_{1} \not\in F \land q_{2} \in F)$
			\item Taktiez plat ize $\forall q_{1}, q_{2} \in Q \forall k \ge 1: q_{1} \overset{k}{\equiv} q_{2} \Leftrightarrow
				(q_{1} \overset{k-1}{\equiv} q_{2}) \land (\forall a \in \Sigma: \delta(q_{1}, a) \overset{k-1}{\equiv} \delta(q_{2}, a))$
			\item $\overset{0}{\equiv}$ je ekvivalenciou urcujucou rozklad $\{F, Q\setminus F\}$
			\item Pokial $q_{1} \overset{k+1}{\equiv} q_{2} = q_{1} \overset{k}{\equiv} q_{2}$ tak z definicie $\equiv$ plynie, ze
				aj $\overset{k}{\equiv} = \overset{k+1}{\equiv} = \overset{k+2}{\equiv} ... = \equiv$
			\item Ekvivalencia $\overset{k+1}{\equiv}$ teda vytvara zjemnenie rozkladu na ekvivalencii $\overset{k}{\equiv}$
			\item Hladana ekvivalencia je tym padom $\overset{k}{\equiv}$
			\item Pretoze $F$ obsahuje najviac $n - 1$ prvkov, tak dokazeme vykonat najviac $n - 2$ zjemneni $\overset{0}{\equiv}$
		\end{itemize}
\end{enumerate}

\textbf{Algoritmus}
\begin{flalign*}
	&i := 0 & \\
	&\overset{0}{\equiv} = \{(p,q) \pipesep p \in F \Leftrightarrow q \in F\} & \\
	&\texttt{repeat} & \\
	&\hspace{1cm} \overset{i+1}{\equiv} = \{(p,q) \pipesep p \overset{i}{\equiv} q
		\land \forall a \in \Sigma : \delta(p,a) \overset{i}{\equiv} \delta(q,a)\} & \\
	&\hspace{1cm} i := i + 1 & \\
	&\texttt{until }\overset{i}{\equiv} \not= \overset{i-1}{\equiv} & \\
	&Q' = Q / \overset{i}{\equiv} & \\
	&\forall p,q \in Q \forall a \in \Sigma: \delta'([p],a) = [q] \Leftrightarrow \delta(p,a) = q & \\
	&q_{0}' = [q_{0}] & \\
	&F' = \{[q] \pipesep q \in F\} &
\end{flalign*}

\subsection*{Regularne mnoziny}
Nech $\Sigma$ je konecna abeceda. Regularnu mnozinu nad $\Sigma$ definujeme rekurzivne takto:
\begin{itemize}
	\item $\varnothing$ je regularna mnozina nad $\Sigma$
	\item $\{\varepsilon\}$ je regularna mnozina nad $\Sigma$
	\item $\{a\}$ je regularna mnozina nad $\Sigma$ pre vsetky $a \in \Sigma$
	\item Ak su $P$ a $Q$ regularne mnoziny nad $\Sigma$, potom tiez
		\begin{itemize}
			\item $P \cup Q$
			\item $P.Q$
			\item $P^{*}$
		\end{itemize}
		su taktiez regularne mnoziny nad $\Sigma$
	\item Ziadne ine mnoziny, nez tie ktore je mozne ziskat podla nasledujucich pravidiel, nie su regularne mnoziny
\end{itemize}

\subsection*{Regularne vyrazy}
Regularne vyrazy a regularne mnoziny, ktore oznacuju su definovane nasledovne:
\begin{itemize}
	\item $\varnothing$ je regularny vyraz oznacujuci regularnu mnozinu $\varnothing$
	\item $\varepsilon$ je regularny vyraz oznacujuci regularnu mnozinu $\{\varepsilon\}$
	\item $a$ je regularny vyraz oznacujuci regularnu mnozinu $\{a\}$ pre vsetky $a \in \Sigma$
	\item Ak su $p,q$ regularne vyrazy oznacujuce regularne mnoziny $P,Q$, potom
		\begin{itemize}
			\item $(p + q)$ je regularny vyraz oznacujuci regularnu mnozinu $P \cup Q$
			\item $(p.q)$ je regularny vyraz oznacujuci regularnu mnozinu $P.Q$
			\item $(p^{*})$ je regularny vyraz oznacujuci regularnu mnozinu $P^{*}$
		\end{itemize}
	\item Ziadne ine regularne vyrazy nad $\Sigma$ neexistuju
\end{itemize}

\subsection*{Kleeneho algebra}
Kleeneho algebra pozostava z jednej mnoziny, dvoch binarnych operacii $+, \cdot$, unarnej operacie $*$ a
dvoch konstant $0$ a $1$. Operacie splnuju vela roznych axiomov, ktore si sem pisat urcite nebudem. :)

Usporiadanie $\le$ definovane ako $a \le b \overset{def}{\Leftrightarrow} a + b = b$

\subsection*{Rovnice nad regularnymi vyrazmi}
Rovnice, ktorej zlozkami su koeficienty a nezname, ktore reprezentuju regularne vyrazy.
\textbf{Najmensi pevny bod} - Rovnica nad regularnymi vyrazmi nad abecedou $\{a,b\}$ $X = aX + b$ ma riesenie
$X = a^{*}b$. Je to tzv. najmensie riesenie, alebo aj najmensi pevny bod.

\subsection*{Regularne mnoziny a jazyky typu 3}
Jazyk $L$ je regularna mnozina prave ked je jazyk $L$ jazyk typu 3.
\begin{equation*}
	\mathcal{L}_{R} = \mathcal{L}_{3}
\end{equation*}

\emph{Dokaz} - kazdu regularnu mnozinu je mozne generovat gramatikou typu 3. Najprv $\mathcal{L}_{R} \subseteq \mathcal{L}_{3}$
\begin{itemize}
	\item $\varnothing - (\{S\},\Sigma,\varnothing,S)$
	\item $\varepsilon - (\{S\},\Sigma,\{S \to \varepsilon\},S)$
	\item $\{a\}\text{ pre kazde }a \in \Sigma - (\{S\},\Sigma,\{S \to a\},S)$
	\item Majme 2 gramatiky $G_{1} = (S_{1}, \Sigma_{1}, P_{1}, S_{1})$ a $G_{1} = (S_{1}, \Sigma_{1}, P_{1}, S_{1})$
	\item $L(G_{1}) \cup L(G_{2}) - G_{3} = (S_{3} \cup N_{1} \cup N_{2}, \Sigma_{1} \cup \Sigma_{2}, \{S_{3} \to S_{1} | S_{2}\} \cup P_{1} \cup P_{2}, S_{3})$
	\item $L(G_{1}).L(G_{2}) - G_{4} = (N_{1} \cup N_{2}, \Sigma_{1} \cup \Sigma_{2}, P_{4}, S_{1})$
		\begin{itemize}
			\item Ak $A \to xB \in P_{1}$ tak $A \to xB \in P_{4}$
			\item Ak $A \to x \in P_{1}$ tak $A \to xS_{2} \in P_{4}$
			\item Ak $A \to \alpha \in P_{2}$ tak $A \to \alpha \in P_{4}$
		\end{itemize}
	\item $L(G_{1})^{*} - G_{5} = (\{S_{5}\} \cup N_{1}, \Sigma_{1}, P_{5}, S_{5})$
		\begin{itemize}
			\item $S_{5} \to S_{1}|\varepsilon \in P_{5}$
			\item Ak $A \to xB \in P_{1}$ tak $A \to xB \in P_{5}$
			\item Ak $A \to x \in P_{1}$ tak $A \to xS_{5} \in P_{5}$
		\end{itemize}
\end{itemize}

\emph{Dokaz} - $\mathcal{L}_{3} \subseteq \mathcal{L}_{R}$ - kazdy jazyk generovany gramatikou typu 3 je regularnou mnozinou.
Ak mame jazyk typu 3, tak prenho mozme zostavit ekvivalentny konecny automat. Z tohto automatu mozeme zostavit sustavu rovnic
nad regularnymi vyrazmi a dostavame regularne vyrazy oznacujuce regularne mnoziny.

\subsection*{Rozsirene konecne automaty}
RKA je patica $M = (Q, \Sigma, \delta, s, F)$
\begin{itemize}
	\item $Q$ - konecna mnozina stavov
	\item $\Sigma$ - konecna vstupna abeceda
	\item $\delta$ - prechodova funkcia $\delta: Q \times (\Sigma \cup \{\varepsilon\}) \to 2^{Q}$
	\item $s$ - pociatocny stav, $s \in Q$
	\item $F$ - mnozina koncovych stavov, $F \subseteq Q$
\end{itemize}

\subsection*{$\varepsilon$-uzaver}
\begin{equation*}
	\varepsilon\text{-uzaver}(q) = \{r \in Q \pipesep \exists w \in \Sigma^{*}: (q,w) \vdash^{*} (r,w)\}
\end{equation*}
\begin{equation*}
	\varepsilon\text{-uzaver}(T) = \underset{s \in T}{\bigcup}\varepsilon\text{-uzaver}(s)
\end{equation*}
Zavedieme relaciu $\overset{\varepsilon}{\to}$
\begin{equation*}
	\forall q_{1}, q_{2} \in Q: q_{1} \overset{\varepsilon}{\to} q_{2} \overset{def}{\Leftrightarrow} q_{2} \in \delta(q_{1}, \varepsilon)
\end{equation*}
Potom mozeme alternativne zapisat
\begin{equation*}
	\varepsilon\text{-uzaver}(q) = \{r \in Q \pipesep q \overset{\varepsilon}{\to}^{*} r\}
\end{equation*}

\subsection*{Prevod RKA na DKA}
\begin{flalign*}
	&Q' = 2^{Q} \setminus \{\varnothing\} & \\
	&q_{0}' = \varepsilon\text{-uzaver}(q_{0}) & \\
	&\delta': Q' \times \Sigma \to Q' & \\
	&\hspace{1cm} \forall T \in Q' \forall a \in \Sigma: \bar{\delta}(T,a) = \underset{q \in T}{\bigcup}\delta(q,a) & \\
	&\hspace{1cm} \text{Ak }\bar{\delta}(T,a) \not= \varnothing\text{ tak }\delta'(T,a) = \varepsilon\text{-uzaver}(\bar{\delta}(T,a)) & \\
	&F' = \{T \pipesep T \in Q' \land T \cap F \not= \varnothing\} &
\end{flalign*}

\subsection*{Prevod RV na RKA}
\begin{enumerate}
	\item Rozlozit regularny vyraz na primitivne zlozky
	\item Zacneme zostrojovat KA pre jednotlive zlozky, nebudem to tu rozpisovat, je mozne to najst v slidoch, ale aj tak to kazdy pozna
\end{enumerate}

\subsection*{Vlastnosti regularnych jazykov}
\subsection*{Konecne jazyky}
Kazdy konecny jazyk je regularny.

\emph{Dokaz} - Nech $L = \{w_{1}, w_{2}, ..., w_{n}\}$. Potom vieme zostrojit gramatiku
$G = (\{S\}, \Sigma, \{S \to w_{1}|w_{2}|...|w_{n}\}, S)$, ktora je podla tvaru pravidiel typu 3.

\subsection*{Pumping lemma}
\begin{equation*}
\exists p > 0: \forall w \in L: |w| \ge p \Rightarrow \exists x,y,z \in \Sigma^{*}: w = xyz \land |xy| \le p \land y \not= \varepsilon \land
	\forall i \ge 0: xy^{i}z \in L
\end{equation*}

\emph{Dokaz} - Nech $L = L(M) = (Q, \Sigma, \delta, q_{0}, F)$ pre ktory plati, ze $|Q| = n$. Majme
retazec $w \in L$, pre ktory plati $|w| \ge n$. Na prijatie takehoto retazca musi automat spravit minimalne
$n$ prechodov medzi $n + 1$ konfiguraciami. Potom je zjavne, ze automat musi prejst minimalne
2 krat cez rovnaky stav pri prijmani tohto retazca. Je tiez zjavne, ze k tomuto opakovaniu musi prist
po nanajvys $n$ znakoch retazca. Rozdelme $w$ na casti $xyz$, tak ako vravi pumping lemma.
Hovorime, ze existuje taky stav $r \in Q$ a take $k, 0 < k \le n$, ze plati
\begin{equation*}
	(q_{0}, xyz) \vdash^{*} (r, yz) \vdash^{k} (r, z) \vdash^{*} (q_{F}, \varepsilon), q_{F} \in F
\end{equation*}
Z toho vyplyva, ze existuje postupnost konfiguracii
\begin{equation*}
	(q_{0}, xy^{i}z) \vdash^{*} (r, y^{i}z) \vdash^{+}(r, y^{i - 1}z) \vdash^{+} ... \vdash^{+} (r, z) \vdash^{*} (q_{F}, \varepsilon), q_{F} \in F
\end{equation*}
Z toho je zjavne, ze $xy^{i}z \in L$ pre lubovolne $i \ge 0$.

\subsection*{Myhill-Nerodova veta}
3 ekvivalentne tvrdenia:
\begin{enumerate}
	\item Jazyk $L$ je prijimany DKA
	\item Jazyk $L$ je zjednotenim niektorych tried relacie ekvivalencie $\sim$ s konecnym indexom, ktora je pravou kongruenciou
	\item Relacia ekvivalencie $\sim_{L}$ ma konecny index
\end{enumerate}

\textbf{Prava kongruencia} - $\forall u,v,w \in \Sigma^{*}: u \sim v \Rightarrow uw \sim vw$

\textbf{Prefixova ekvivalencia} - $\forall u,v \in \Sigma^{*}: u \sim_{L} v \overset{def}{\Leftrightarrow} \forall w \in \Sigma^{*}: uw \in L \Leftrightarrow vw \in L$

\emph{Dokaz}
\begin{itemize}
	\item $1 \Rightarrow 2$
		\begin{itemize}
			\item Majme DKA $(Q,\Sigma,\delta,q_{0},F)$
			\item Definujme vseobecnu prechodovu funkciu $\delta': Q \times \Sigma^{*} \to Q$ nasledovne
			\item $\forall q,p \in Q \forall w \in \Sigma^{*}: \delta'(q,w) = p \Leftrightarrow (q,w) \vdash^{*} (p, \varepsilon)$
			\item Definujme relaciu ekvivalencie $\sim$ nasledovne $\forall u,v \in \Sigma^{*}: u \sim v \Leftrightarrow \delta'(q_{0}, u) = \delta'(q_{0}, v)$
			\item Tato relacia je ekvivalencia (reflexivna, symetricka, tranzitivna)
			\item Je aj pravou kongruenciou
				\begin{equation*}
					\forall w \in \Sigma^{*}: \delta'(q_{0}, uw) = \delta'(\delta'(q_{0}, u), w) = \delta'(\delta'(q_{0}, v), w) = \delta'(q_{0}, vw)
				\end{equation*}
			\item Ma konecny index - pocet stavov
			\item Jazyk $L$ je zjednotenim niektorych tried ekvivalencie, a to tych, ktore odpovedaju stavom v $F$
		\end{itemize}
	\item $2 \Rightarrow 3$ (Vojnar)
		\begin{itemize}
			\item Majme $u,v \in \Sigma^{*}$ take, $u \sim_{L} v$
			\item Ukazeme, ze $\sim_{L}$ je prava kongruencia, teda $uw \sim_{L} vw$
			\item Aby platilo $uw \sim_{L} vw$, tak musi platit ze $\forall w' \in \Sigma^{*}: uww' \in L \Leftrightarrow vww' \in L$
			\item Nakolko je vsak $\sim_{L}$ prefixova ekvivalencia, tak vieme ze $\forall w'' \in \Sigma^{*}: uw'' \in L \Leftrightarrow vw'' \in L$
			\item Vzhladom na uzavretost $\Sigma^{*}$ voci $\cdot$, mozeme predpokladat, ze $ww' \in \Sigma^{*}$ odpoveda $w'' \in Sigma^{*}$ a teda aj plati
				ze $\forall w' \in \Sigma^{*}: uww' \in L \Leftrightarrow vww' \in L$
		\end{itemize}
	\item $2 \Rightarrow 3$ (Slidy)
		\begin{itemize}
			\item Majme $u,v \in \Sigma^{*}$ take, ze $u \sim v$. Ukazeme ze aj $u \sim_{L} v$.
			\item Ak ma platit $u \sim_{L} v$, tak musi platit $\forall w \in \Sigma^{*}: uw \Leftrightarrow vw$
			\item Vieme, ale ze $\sim$ je prava kongruancia, teda plati $uw \sim vw$ a nakolko je jazyk $L$
				zjednotenim niektorych tried rozkladu $\Sigma^{*}\setminus\sim$, tak plati aj
				$uw \in L \Leftrightarrow vw \in L$
			\item Pokial ma $\sim$ konecny index, tak aj $\sim_{L}$ ma konecny index
			\item Plati pritom, ze $\sim \subseteq \sim_{L}$
			\item $\sim_{L}$ je najvacsia prava kongruancia na $\Sigma^{*}$
		\end{itemize}
	\item $3 \Rightarrow 1$
		\begin{itemize}
			\item Zostrojime DKA $M = (Q,\Sigma,\delta,q_{0},F)$, tak aby $L = L(M)$
			\item $Q = \Sigma^{*}\setminus\sim_{L}$
			\item $\forall w \in \Sigma^{*} \forall a \in \Sigma: \delta([w], a) = [wa]$
			\item $q_{0} = [\varepsilon]$
			\item $F = \{[w]\pipesep w \in L\}$
			\item Tento automat odpoveda minimalnemu DKA
		\end{itemize}
\end{itemize}

\subsection*{Uzaverove vlastnosti}
Trieda regularnych jazykov je uzavrena voci zjednoteniu, konkatenacii a iteracii.

Trieda regularnych jazykov tvori \textbf{Booleovu algebru} - $(\mathcal{L}_{3}, \cup, \cap, ', \Sigma^{*}, \varnothing)$
\begin{enumerate}
	\item $\Sigma^{*} \in \mathcal{L}_{3}$ a aj $\varnothing \in \mathcal{L}_{3}$. Uzavretost voci $\cup$ je zrejma.
	\item \emph{Dokaz uzavretosti komplementu} - doplnku
	\begin{itemize}
		\item Majme abecedu $\Sigma$ a nejaku abecedu $\Delta \subseteq \Sigma$
		\item Zostrojme automat $M = (Q,\Delta,\delta,q_{0},F)$, ktory prijma jazyk $L(M)$ nad abecedou $\Delta$
		\item Automat prijmajuci komplement $L(M)$ voci $\Delta^{*}$ oznacme ako $M'$
		\item $M' = (Q,\Delta,\delta,q_{0},Q \setminus F)$ prijma jazyk $L(M')$, co je komplement ku $L(M)$ voci $\Delta^{*}$
		\item Komplement voci $\Sigma^{*}$ je potom jazyk $\overline{L}$
		\item $\overline{L} = L(M') \cup \Sigma^{*}(\Sigma \setminus \Delta)\Sigma^{*}$
	\end{itemize}
	\item \emph{Dokaz uzavretosti voci $\cap$}
	\begin{itemize}
		\item Plynie z DeMorganovych zakonov
		\item $L_{1} \cap L_{2} = \overline{\overline{L_{1} \cap L_{2}}} = \overline{\overline{L_{1}} \cup \overline{L_{2}}}$
	\end{itemize}
\end{enumerate}

\subsection*{Rozhodnutelne problemy}
\begin{enumerate}
	\item \textbf{Neprazdnost} (rozhodnutelny) - $L(M) \not= \varnothing \Leftrightarrow \exists q \in Q : ((\exists w \in \Sigma^{*}: (q_{0}, w) \vdash^{*} (q, \varepsilon)) \land q \in F)$
	\item \textbf{Nalezitost} (rozhodnutelny) - $w \in L \Leftrightarrow \exists q \in Q: (q_{0}, w) \vdash^{*} (q, \varepsilon) \land q \in F$
	\item \textbf{Ekvivalencia} (rozhodnutelny) - $L(G_{1}) = L(G_{2}) resp. L_{1} = L_{2}?$
		\begin{itemize}
			\item Nech $M_{1}$ a $M_{2}$ su KA prijmajuce $L_{1}$ resp. $L_{2}$
			\item Zostrojime automat $M = (Q_{1} \cup Q_{2}, \Sigma_{1} \cup \Sigma_{2}, \delta_{1} \cup \delta_{2}, q_{0}^{1}, F_{1} \cup F_{2})$
			\item Vypocitame relaciu nerozlisitelnosti $\equiv$
			\item $L_{1} = L_{2} \Leftrightarrow q_{0}^{1} \equiv q_{0}^{2}$
		\end{itemize}
\end{enumerate}

\section{Bezkontextove jazyky}
\textbf{Derivacny strom}
\begin{itemize}
	\item Nech $\alpha$ je vetna forma generovana v gramatike $G$ a nech $S \Rightarrow w_{1} \Rightarrow w_{2} \Rightarrow ... \Rightarrow \alpha$
		je jej derivacia. Derivacny strom je stromove znazornenie derivacnych krokov v gramatike $G$. Jednotlive uzly stromu maju nasledujuce vlastnosti:
	\item Jednotlive uzly stromu su ohodnotene mnozinou $N \cup \Sigma$. Koren stromu je ohodnoteny neterminalom $S$.
	\item Priamej derivacii $\lambda A\nu \Rightarrow \lambda X_{1}X_{2}...X_{n}\nu$, na zaklade prepisovacieho pravidla $A \to X_{1}X_{2}...X_{n}$
		odpoveda $n$ hran, ktore smeruju od rodicovskeho uzlu $A$ k uzlom $X_{i}, 1 \le i \le n$.
	\item Ohodnotenie koncovych uzlov derivacneho stromu z lava do prava vytvara vetnu formu
\end{itemize}

\textbf{Lava a prava derivacia}
\begin{itemize}
	\item Nech je $\lambda A\nu$ vetna forma taka, ze $\lambda \in \Sigma^{*}, A \in N, \nu \in (N \cup \Sigma^{*})$ a nech v gramatike $G$ existuje
		pravidlo $A \to \alpha$. Derivaciu $\lambda A\nu \Rightarrow \lambda\alpha\nu$ nazyva lavou derivaciou. Analogicky prava derivacia.
	\item Slidy - Nech $S \Rightarrow \alpha_{1} \Rightarrow \alpha_{2} \Rightarrow ... \Rightarrow \alpha$ je derivacia vetnej formy $\alpha$.
		Ak bol v kazdom retazci prepisan najlavejsi neterminal, potom tuto derivaciu nazyvame lavou (pravou) derivaciou vetnej formy $\alpha$.
\end{itemize}

\textbf{Fraza vetnej formy}
Nech $G$ je gramatika a nech retazec $\lambda = \alpha\beta\gamma$ je vetna forma. Podretazec $\beta$ sa nazyva frazou vetnej forma vzhladom k neterminalu $A$, pokial plati
$S \Rightarrow^{*} \alpha A\gamma$ a $A \Rightarrow^{+} \beta$. $\beta$ je jednoduchou frazou vetnej formy pokial $A \Rightarrow \beta$.

\textbf{Viacznacnost gramatik}
Nech $G$ je gramatika a $w$ je veta. Hovorime, ze veta $w$ je viacznacna, ak existuje viac derivacnych stromov, ktore generuju tuto vetu.
Hovorime, ze gramatika je viacznacna, pokial generuje aspon 1 viacznacnu vetu. Inak hovorime o jednoznacnej gramatike. Jazyky, ktore je mozne
generovat viacznacnou gramatikou, ale nie jednoznacou sa nazyvaju jazyky s inherentnou viacznacnostou.

\subsection*{Transofrmacie bezkontextovych gramatik}
\textbf{Ekvivalentne gramatiky}
\begin{itemize}
	\item Nech $G = (N, \Sigma, P, S)$. Pokial je $A \to \alpha B\beta \in P$ a $B \to \gamma_{1}|\gamma_{2}|...|\gamma_{n}$ su vsetky $B$-pravidla v $P$,
		potom gramatika $G' = (N, \Sigma, P', S)$, kde
		\begin{equation*}
			P' = P \setminus \{ A \to \alpha B\beta\} \cup \{A \to \alpha\gamma_{1}\beta|\alpha\gamma_{2}\beta|...|\alpha\gamma_{n}\beta\}
		\end{equation*}
		je ekvivalentna s gramatikou $G$
\end{itemize}

\textbf{Nedostupne a zbytocne symboly}
Nech je $G$ gramatika a $X \in (N \cup \Sigma)$ je symbol. Symbol moze byt
\begin{itemize}
	\item \textbf{Nedostupny} - Symbol $X$ je nedostupny pokial neexistuje derivacia $S \Rightarrow^{*} \alpha X\beta$ pre nejake $\alpha, \beta \in (\Sigma \cup N)^{*}$
	\item \textbf{Zbytocny} - Symbol $X$ je zbytocny, pokial neexistuje derivacia $S \Rightarrow^{*} \alpha X\beta \Rightarrow^{*} w$ pre nejake $\alpha, \beta \in (\Sigma \cup N)^{*},
		w \in \Sigma^{*}$
\end{itemize}

\textbf{Neterminaly generujuce terminalne retazce}
\begin{itemize}
	\item Vstup: $G = (N, \Sigma, P, S)$
	\item Vystup: $N_{i} = \{A \pipesep A \Rightarrow^{+} w, w \in \Sigma^{*}\}$
	\begin{flalign*}
		&i := 1 & \\
		&N_{0} = \varnothing & \\
		&\texttt{repeat} & \\
		&\hspace{1cm} N_{i} = N_{i-1} \cup \{A \pipesep A \to \alpha \in P \land \alpha \in (N_{i-1} \cup \Sigma)^{*}\} & \\
		&\hspace{1cm} i := i + 1 & \\
		&\texttt{until }N_{i} = N_{i-1} & \\
	\end{flalign*}
\end{itemize}

\textbf{Dostupne symboly}
\begin{itemize}
	\item Vstup: $G = (N, \Sigma, P, S)$
	\item Vystup: $V_{i} = \{X \pipesep S \Rightarrow^{*} \alpha X\beta, \alpha, \beta \in (N \cup \Sigma)^{*}\}$
	\begin{flalign*}
		&i := 1 & \\
		&V_{0} = \{S\} & \\
		&\texttt{repeat} & \\
		&\hspace{1cm} V_{i} = V_{i-1} \cup \{X \pipesep A \to \alpha X\beta \in P \land A \in V_{i-1}\} & \\
		&\hspace{1cm} i := i + 1 & \\
		&\texttt{until }V_{i} = V_{i-1} & \\
	\end{flalign*}
\end{itemize}

\textbf{Odstranenie zbytocnych symbolov}
\begin{itemize}
	\item Vstup: $G = (N, \Sigma, P, S)$
	\item Vystup: $G' = (N', \Sigma', P', S)$
	\item Vypocitaj $N_{i}$ podla algoritmu na vypocet neterminalov generujucich terminalne retazce
	\item Zostroj $\overline{G} = (N_{i} \cup \{S\}, \Sigma, \overline{P}, S)$
	\item $\overline{P} = \{A \to \alpha \pipesep A \to \alpha \in P \land A \in N_{i} \land \alpha (N_{i} \cup \Sigma)^{*}\}$
	\item Vypocitaj $V_{i}$ podla algoritmu na vypocet dostupnych symbolov
	\item $N' = N_{i} \cap V_{i}$
	\item $\Sigma' = \Sigma \cap V_{i}$
	\item $P' = \{A \to \alpha \pipesep A \to \alpha \in \overline{P} \land A \in N_{i} \cap V_{i} \land \alpha \in V_{i}^{*}\}$
\end{itemize}

\textbf{Odstranenie $\varepsilon$-pravidiel}
\begin{itemize}
	\item Vstup: $G = (N, \Sigma, P, S)$
	\item Vystup: $N_{\varepsilon} = \{A \pipesep A \Rightarrow^{+} \varepsilon\}$
	\begin{flalign*}
		&i := 1 & \\
		&N_{\varepsilon}^{0} = \varnothing & \\
		&\texttt{repeat} & \\
		&\hspace{1cm} N_{\varepsilon}^{i} = N_{\varepsilon}^{i-1} \cup \{A \pipesep A \to \alpha \in P \land \alpha \in (N_{\varepsilon}^{i-1})^{*}\} & \\
		&\hspace{1cm} i := i + 1 & \\
		&\texttt{until }N_{\varepsilon}^{i} = N_{\varepsilon}^{i-1} & \\
	\end{flalign*}
\end{itemize}

\textbf{Odstranenie jednoduchych pravidiel}
\begin{itemize}
	\item Vstup: $G = (N, \Sigma, P, S)$
	\item Vystup: $G' = (N, \Sigma, P', S)$
	\item Pre vsetky $A \in N$ vypocitaj mnozinu $N_{A} = \{X \pipesep A \Rightarrow^{*} X\}$
	\item Pre vsetky $X \in N_{A}$, pokial $X \to \alpha \in P$ a $\alpha \not\in N$, tak pridaj $A \to \alpha$ do $P'$
\end{itemize}

\textbf{Cyklus}

Gramatika $G$ obsahuje cyklus pokial $A \in N: A \Rightarrow^{+} A$. Pokial gramatika obsahuje cyklus, tak je viacznacna.

\emph{Dokaz} - Pokial existuje derivacia $S \Rightarrow^{*} \alpha A\beta \Rightarrow^{+} w$, tak existuje aj derivacia
$S \Rightarrow^{*} \alpha A\beta \Rightarrow^{+} \alpha A\beta \Rightarrow^{+} w$. Tymto derivaciam prisluhuje viac
derivacnych stromov.

\textbf{Vlastna gramatika}

Gramatika bez zbytocnych symbolov, $\varepsilon$-pravidiel a cyklov

\textbf{Odstranenie lavej rekurzie}

Nech gramatika $G$ ma pravidla v tvare
\begin{equation*}
	A \to A\gamma_{1} | A\gamma_{2} | ... | A\gamma_{n} | \beta_{1} | \beta_{2} | ... | \beta_{n}
\end{equation*}
Odstranenie lavej rekurzie je mozne nasledovnym postupom
\begin{equation*}
	A \to \beta_{1}A' | \beta_{2}A' | ... | \beta_{n}A' | \beta_{1} | \beta_{2} | ... | \beta_{n}
\end{equation*}
\begin{equation*}
	A' \to \gamma_{1}A' | \gamma_{2}A' | ... | \gamma_{n}A' | \gamma_{1} | \gamma_{2} | ... | \gamma_{n}
\end{equation*}

\textbf{Odstranenie nepriamej lavej rekurzie}

Aplikujeme vetu o ekvivalentnych gramatikach pokial sa nedostaneme k jednoduchej lavej rekurzii. Potom len odstranime jednoduchu
lavu rekurziu.

\subsection*{Normalne formy bezkontextovych gramatik}

\textbf{CNF - Chomskeho Normalna Forma}
\begin{itemize}
	\item BKG $G = (N, \Sigma, P, S)$ je v CNF, pokial ma kazde pravidlo v $P$ jeden z tychto tvarov:
		\begin{enumerate}
			\item $A \to BC$ kde $A,B,C \in N$
			\item $A \to a$ kde $A \in N, a \in \Sigma$
			\item Ak $\varepsilon \in L(G)$ tak $S \to \varepsilon$ je jedine $\varepsilon$-pravidlo
		\end{enumerate}
	\item Pre kazdu gramatiku $G$ existuje gramatika $G'$, ktora je v CNF
		\begin{itemize}
			\item Pravidla podla tvaru $1,2,3$ prekopirujeme do $P'$
			\item Pre pravidla v tvare $A \to X_{1}X_{2}...X_{n}$
				zavedieme pravidla $A \to X_{1}\langle X_{2}...X_{n}\rangle$ a $\langle X_{2},X_{3}...X_{n}\rangle \to X_{2}\langle X_{3}...X_{n}\rangle$ atd.
				Pre $n = 2$ zavedieme $\langle X_{n-1}X_{n}\rangle \to X_{n-1}X_{n}$.
				V pripade ze $X_{i} \in \Sigma$, tak zaviedme neterminal $X_{i}'$, ktorym nahradime povodny $X_{i}$ a vytvorime pravidlo
				$X_{i}' \to X_{i}$.
			\item Pre pravidla tvaru $A \to X_{1}X_{2}$, v pripade ze $X_{i}$ tak zaviedme neterminal $X_{i}'$,
				ktorym nahradime povodny $X_{i}$ a vytvorime pravidlo $X_{i}' \to X_{i}$.

		\end{itemize}
\end{itemize}

\textbf{GNF - Greibachovej Normalna Forma}
\begin{itemize}
	\item BKG $G = (N, \Sigma, P, S)$ je v GNF, pokial ma kazde pravidlo v $P$ nasledujuci tvar
	\begin{itemize}
		\item $A \to a\alpha$ kde $a \in \Sigma, \alpha \in N^{*}$
	\end{itemize}
	\item Odstranime lavu rekuziu a $\varepsilon$ pravidla
	\item Nech $G = (N, \Sigma, P, S)$ je BKG bez lavej rekurzie, potom existuje usporiadanie $\prec$, take ze
		pokial existuje pravidlo $A \to B\alpha$, potom $A \prec B$
	\item Najdeme toto usporiadanie a v opacnom poradi, ako je dane usporiadanie zacneme substituovat neterminaly
	\item Pre kazdy terminal $a$ nachadzajuci sa na inej ako prvej pozicii na pravej strane pravidla vytvorime pravidlo
		$a' \to a$ a nahradime neprve vyskyty $a$ za $a'$
\end{itemize}

\subsection*{Zasobnikove automaty}
$P = (Q,\Sigma,\Gamma,\delta,q_{0},Z_{0},F)$
\begin{enumerate}
	\item $Q$ -- konecna mnozina stavov
	\item $\Sigma$ -- konecna vstupna abeceda
	\item $\Gamma$ -- konecna zasobnikova abeceda
	\item $\delta$ -- prechodova funkcia $Q \times (\Sigma \cup \{\varepsilon\}) \times \Gamma \to 2^{Q \times \Gamma^{*}}$
	\item $q_{0} \in Q$ -- pociatocny stav
	\item $Z_{0} \in \Gamma$ -- pocitacony zasobnikovy symbol
	\item $F \subseteq Q$ -- mnozina koncovych stavov
\end{enumerate}

\textbf{Konfiguracia ZA}

Konfiguracia ZA $(q,w,\alpha) \in Q \times \Sigma^{*} \times \Gamma^{*}$

\textbf{Prechod ZA}

$(q,w,\beta) \vdash (q',w',\beta') \overset{def}{\Leftrightarrow} w = aw' \land \beta' = \gamma\alpha \land \beta = Z\alpha \land (q',\gamma) \in \delta(q,a,Z)$

\textbf{Jazyk prijmany ZA}

$L(P) = \{w \in \Gamma^{*} \pipesep (q_{0},w,Z_{0}) \vdash^{*} (q_{F},\varepsilon,\gamma) \land q_{F} \in F\}$

\textbf{Rozsireny ZA}

$\delta: Q \times (\Gamma \cup \{\varepsilon\}) \times \Gamma^{*} \to 2^{Q \times \Gamma^{*}}$

\textbf{Ekvivalencia ZA a RZA}

Nech $P = (Q,\Sigma,\Gamma,q_{0},z_{0},F)$ je RZA, potom existuje ZA $P'$ taky, ze $L(P) = L(P')$

\emph{Dokaz} \\
Polozme $m = max\{|\alpha| \pipesep \delta(q,a,\alpha) \not= \varnothing, q \in Q, a \in \Sigma \cup \{\varepsilon\}, \alpha \in \Gamma^{*}\}$
TODO VRATIT SA K TOMUTO

\textbf{ZA prijamjuci s vyprazdnenim zasobniku}

Pokial

$L(P) = \{w \in \Sigma^{*} \pipesep (q_{0},w,z_{0}) \vdash^{*} (q,\varepsilon,\varepsilon)\}$

Ku kazdemu ZA (RZA), ktory prijma prechodom do koncoveho stavu, existuje ZA (RZA), ktory prijma vyprazdnenim zasobniku.

Automat $P'$ bude simulovat $P$ tak, ze pri prechode do jedneho z koncovych stavov prejde do specialneho stavu $q_{\varepsilon}$, ktory sposobi vyprazdnenie zasobniku.
Aby sme ale zamedzili nechcenemu vyprazdneniu, tak na dno zasobnika umiestnime specialny symbol. Tento symbol moze vytiahnut len $q_{\varepsilon}$.

\textbf{Ekvivalencia BKJ a jazykov prijmanych ZA - $\mathcal{L}_{2} = \mathcal{L}_{P}$}

$\mathcal{L}_{2} \subseteq \mathcal{L}_{P}$

Ku gramatike $G$ zostrojime ZA $P'$ tak, aby modeloval syntakticku analyzu zhore nadol.
\begin{itemize}
	\item Ak $A \to \alpha \in P$, tak $(q,\alpha) \in \delta(q,\varepsilon,A)$
	\item Pre kazde $a \in \Sigma$, pridaj $(q,\varepsilon) \in \delta(q,a,a)$
\end{itemize}

Ku gramatike $G$ zostrojime RZA $P'$ tak aby modeloval syntakticku analyzu zdola nahor.
\begin{itemize}
	\item Ak $A \to \alpha \in P$, tak $(q,A) \in \delta(q,\varepsilon,\alpha^{R})$
	\item Pre kazde $a \in \Sigma$, pridaj $(q,a) \in \delta(q,a,\varepsilon)$
	\item $(r,\varepsilon) \in \delta(q,\varepsilon,S\#)$
\end{itemize}

$\mathcal{L}_{P} \subseteq \mathcal{L}_{2}$

TODO

\subsection*{Vlastnosti bezkontextovych jazykov}

\textbf{Pumping lemma}

Nech $L$ je BKJ,

$\exists p > 0: \forall z \in L: |z| \ge p \Rightarrow z = uvwxy \land |vwx| \le p \land vx \not= \varepsilon \land \forall i: uv^{i}wx^{i}y \in L$

\emph{Dokaz}

Majme BKG $G$ v CNF
\begin{itemize}
	\item Nech $A \Rightarrow^{+} w$ je derivacia a $m$ je dlzka maximalnej cesty v derivacnom strome
	\item Potom $|w| \le 2^{m-2}$, lebo $2^{m-2}$ odpoveda maximalnemu poctu listov v tomto binarnom strome
	\item Polozme $k = 2^{N}$ a uvazujme vetu $z, |z| \ge k$
	\item $2^{|N|} \le 2^{m-2} \Rightarrow |N| + 2 \le m$, najdlhsia cesta teda obsahuje az $|N| + 2$ vrcholov,
		jeden z toho je terminal a jeden neterminal $A$ sa tam teda opakuje
	\item $A \Rightarrow^{*} uAy \Rightarrow^{+} uvAxy \Rightarrow^{*} uvwxy = z$
	\item Potom ale aj $uv^{i}wv^{i}y \in L(G)$
\end{itemize}

\textbf{Substitucia jazykov}

$\sigma_{L_{a_{1}},L_{a_{2}},...,L_{a_{n}}}(L) = \{x_{1}x_{2}...x_{m} \pipesep b_{1}b_{2}...b_{m} \land \forall i \in \{1,...,m\}: x_{i} \in L_{b_{i}}\}$

\textbf{Morfizmus jazykov}

$h: \Sigma^{*} \to \Delta^{*}$\\
$\forall w \in \Sigma^{*}, w = a_{1}a_{2}...a_{n} : h(w) = h(a_{1})h(a_{2})...h(a_{n})$\\

\textbf{Uzavretost voci substitucii}

\begin{itemize}
	\item Majme jazyk L generovany $G = (N,\Sigma,P,S)$ a jazyky $L_{a}$ generovane $G_{a} = (N_{a}, \Sigma_{a}, P_{a}, S_{a})$
	\item Zobrazenie $h$ definovane takto:
		\begin{itemize}
			\item $h(A) = A, A \in N$
			\item $h(a) = S_{a}, a \in \Sigma$
		\end{itemize}
	\item Jazyk $\sigma(L) = L'$ bude generovany gramatikou $G'=(N',\Sigma',P',S')$
	\item $N' = N \cup \bigcup N_{a}$
	\item $\Sigma' = \bigcup \Sigma_{a}$
	\item $P' = \{A \to h(\alpha) \pipesep A \to \alpha \in P\} \cup \bigcup P_{a}$
\end{itemize}

\textbf{Uzavretost voci inym operaciam} \\
Majme 2 jazyky $L_{a}$ a $L_{b}$
\begin{itemize}
	\item Uzavretost voci $\cup$ -- substitucia do $\{a,b\}$
	\item Uzavretost voci $\cdot$ -- substitucia do $\{ab\}$
	\item Uzavretost voci $*$ -- substitucia do $\{a\}^{*}$
	\item Uzavretost voci $+$ -- substitucia do $\{a\}^{+}$
	\item Uzavretost voci morfizmu -- vyplyva zo subtitucie
	\item Uzavretost voci prieniku s reg. jazykmi -- konstrukcia ZA, ktory zaroven vykonava ZA a KA
	\item Uzavretost voci inverznemu morfizmu
		\begin{itemize}
			\item Majme jazyk $L \subseteq \Sigma_{2}^{*}$
			\item Idea spociva v konstrukcii $\overline{\Sigma_{1}} = \{\overline{a} \pipesep a \in \Sigma_{1}\}$
			\item Nasledne $\forall c \in \Sigma_{2}: L_{c} = \overline{\Sigma_{1}}^{*}c\overline{\Sigma_{1}}^{*}$
			\item Definujeme jazyk $L_{1} = \{\overline{a}.h(a) \pipesep a \in \Sigma_{1}\}$
			\item Polozime $L_{2} = \sigma(L) \cap L_{1}^{*}$
			\item Zostrojime morfizmus $h1: (\overline{\Sigma_{1}} \cup \Sigma_{2})^{*} \to \Sigma_{1}^{*}$, tak ze $\overline{a}\in \overline{\Sigma_{1}}$ sa zobrazi na $a$ a
				$a \in \Sigma_{2}$ sa zobrazi na $\varepsilon$
			\item $h^{-1}(L) = h1(L_{2})$
			\item $L_{1}$ je jazyk regularny, $\sigma(L)$ je jazyk BKJ (z uzavrenosti substitucie), prienik BKJ a RJ je BKJ, morfizmus je tiez uzavrety, takze inverzny morfizmus tiez
		\end{itemize}
\end{itemize}

\textbf{Neuzavrenost}
\begin{itemize}
	\item Neuzavrenost voci $\cap$ -- $a^{n}.b^{n}.c^{m} \cap a^{m}.b^{n}.c^{n} = a^{n}.b^{n}.c^{n}$
	\item Neuzavrenost voci doplnku -- $L_{1} \cap L_{2} = \overline{\overline{L_{1}} \cup \overline{L_{2}}}$
\end{itemize}

\textbf{Deterministicky ZA}

ZA nazyva deterministicky pokial pre kazde $q \in Q$ a $z \in \Gamma$ plati bud
\begin{itemize}
	\item $|\delta(q,a,z)| \le 1, a \in \Sigma \land \delta(q,\varepsilon,z) = \varnothing$
	\item Alebo $\delta(q,a,z) = \varnothing, a \in \Sigma \land |\delta(q,\varepsilon,z)| \le 1$
\end{itemize}
$L(P)$ sa nazyva deterministicky bezkontextovy jazyk

DZA s vyprazdnenim zasobniku maju mensiu vyjadrovaciu silu ako DZA prechodom do koncoveho stavu. (nemozu prijat
jazyk, ktory obsahuje 2 take retazce ze 1 je prefixom druheho)

\textbf{Deterministicky RZA}
\begin{itemize}
	\item $\forall q \in Q, a \in \Sigma \cup \{\varepsilon\} \forall \gamma \in \Gamma^{*}: |\delta(q,a,\gamma)| \le 1$
	\item Ak $\delta(q,a,\alpha) \not= \varnothing$ a $\delta(q,a,\beta) \not= \varnothing$ a $\alpha \not= \beta$ potom $\alpha$ nie je prefixom $\beta$
		a $\beta$ nie je prefixom $\alpha$
	\item Ak $\delta(q,a,\alpha) \not= \varnothing$ a $\delta(q,\varepsilon,\beta) \not= \varnothing$ potom $\alpha$ nie je prefixom $\beta$
		a $\beta$ nie je prefixom $\alpha$
\end{itemize}

DRZA maju ekvivalentnu silu ako DZA

DZA maju striktne mensiu vyjadrovaciu schopnost ako NZA

\section{Turingove stroje}
\subsection*{Turingov stroj}
$M = (Q,\Sigma,\Gamma,\delta,q_{0},q_{F})$
\begin{itemize}
	\item $Q$ -- konecna mnozina stavov
	\item $\Sigma$ -- konecna vstupna abeceda, $\Delta \not\in \Sigma$
	\item $\Gamma$ -- konecna paskova abeceda, $\Sigma \subset \Gamma, \Delta \in \Gamma$
	\item $\delta$ -- prechodova funkcia, $(Q \setminus \{q_{F}\}) \times \Gamma \to Q \times (\Gamma \times \{L,R\}), L,R \not\in \Gamma$
	\item $q_{0}$ -- pociatocny stav
	\item $q_{F}$ -- prijmajuci stav
\end{itemize}

\textbf{Konfiguracia pasky}

Konfiguracia pasky je $(w,n) \in \{\gamma\Delta^{\omega} \pipesep \gamma \in \Gamma^{*}\} \times \mathbb{N}$

\textbf{Konfiguracia stroja}

Konfiguracia stroja je $(q,w,n) \in Q \times \{\gamma\Delta^{\omega} \pipesep \gamma \in \Gamma^{*}\} \times \mathbb{N}$

\textbf{Prechodova relacia TS}

$\gamma_{n}$ -- n-ty symbol retazca $\gamma$\\
$s^{n}_{b}(\gamma)$ -- substitucia n-teho symbolu v retazci $\gamma$ za symbol $b$ \\
Krok vypoctu $\vdash$
\begin{itemize}
	\item $(q,\gamma,n) \vdash (p,\gamma,n+1)$ pre $(p,R) = \delta(q,\gamma_{n})$, $n \ge 0$
	\item $(q,\gamma,n) \vdash (p,\gamma,n-1)$ pre $(p,L) = \delta(q,\gamma_{n})$, $n > 0$
	\item $(q,\gamma,n) \vdash (p,s^{n}_{b}(\gamma),n)$ pre $(p,b) = \delta(q,\gamma_{n})$, $b \in \Gamma$
\end{itemize}

\textbf{Vypocet TS}

Vypocet TS je postupnost konfiguracii stroja $K_{0}, K_{1}, ...$, taka ze $K_{i} \vdash K_{i+1}$, ktora je bud
\begin{itemize}
	\item Nekonecna
	\item Konecna, s koncovou konfiguraciou $(q,\gamma,n)$, pricom pozname 2 typy zastavenia
		\begin{itemize}
			\item Normalne -- prechod do $q_{F}$
			\item Abnormalne -- prechod z konfiguracie $(q,\gamma,0)$ dolava, alebo nedefinovana $\delta$ pre $q \in Q, a \in \Gamma$
		\end{itemize}
\end{itemize}

\textbf{Jazyk prijmany TS}

Retazec $w \in \Sigma^{*}$ je prijaty strojom $M = (Q,\Sigma,\Gamma,\delta,q_{0},q_{F})$ pokial $M$ z pociatocnej konfiguracie
$(q_{0},\Delta w \Delta^{\omega}, 0)$ prejde do konfiguracie $(q_{F}, \gamma, n)$, pre nejake $\gamma \in \Gamma^{\omega}, n \in \mathbb{N}$.

\textbf{Prijatie specialnou konfiguraciou pasky}

TS prejde z konfiguracie pasky $(\Delta w \Delta^{\omega}, 0)$ do konfiguracie pasky $(\Delta Y \Delta^{\omega}, 0)$\\
Pokial mame TS, ktory prijma prechodom do $q_{F}$, tak mozeme zostrojit TS $M'$, ktory prijme konfiguraciou pasky.

\begin{itemize}
	\item Doplnime uvodnu reziu, ktora konfiguraciu pasky $(\Delta w \Delta^{\omega}, 0)$ zmeni na
		$(\#\Delta w * \Delta^{\omega}, 1)$
	\item Spustime povodny $M$
	\item Vykoname najazd na $*$, nasledne postupujeme smerom dolava a prepisujeme obsah pasky so symbolom $\Delta$
		az pokym nenarazime na $\#$. Potom len zapiseme $Y$ na druhu poziciu -- $\Delta RYL$
\end{itemize}

\textbf{Viacpaskovy TS}

Uvazujeme TS, ktory ma $k$ pasiek a $k$ paskovych abecied $\Gamma_{1}, \Gamma_{2}, ..., \Gamma_{k}$
a prechodovu funkciu $\delta$ ako
\begin{equation*}
	(Q \setminus \{q_{F}\}) \times \Gamma_{1} \times \Gamma_{2} \times ... \times \Gamma_{k} \to Q \times \bigcup \{i\} \times (\Gamma_{i} \cup \{L,R\})
\end{equation*}
Pre kazdy $k$ paskovy TS existuje ekvivalentny 1 paskovy TS
TODO

\textbf{Nedeterministicky TS}

$\delta: (Q \setminus \{q_{F}\}) \times \Gamma \to 2^{Q \times (\Gamma \cup \{L,R\}}$

\subsection*{Jazyky rekurzivne a rekurzivne vycislitelne}

TS je uplny, pokial zastavi pre kazdy svoj vstup \\
Jazyk $L$ sa nazyva
\begin{itemize}
	\item \textbf{Rekurzivny} -- ak $L = L(M)$, pre nejaky uplny TS
	\item \textbf{Rekurzivne vycislitelny} -- ak $L = L(M)$ pre nejaky TS
\end{itemize}
Ak je M uplny, tak $M$ \textbf{rozhoduje} jazyk $L$

\textbf{Rozhodovaci problem} \\
Rozhodovaci problem $P$ moze byt chapany ako funkcia $f_{P}$ s oborom hodnot $\{true, false\}$.\\
Rozhodovaci problem ma specifikovane\\
\begin{itemize}
	\item Definicny obor $A_{P}$ -- instance daneho problemu
	\item Podmnozinou $B_{P} \subseteq A_{P}$, pre ktoru plati ze $B_{P} = \{p \in A_{P} \pipesep f_{P}(p) = true\}$
\end{itemize}

\textbf{Rozhodovanie problemov TS}\\
Nech je problem $P$ specifikovany jazykom $L_{P}$. Nazveme h
\begin{itemize}
	\item \textbf{Rozhodnutelny} -- ak existuje uplny TS M, ktory ho rozhoduje, tj. pokial je $L_{P}$ rekurzivny jazyk
	\item \textbf{Nerozhodnutelny} -- nie je rozhodnutelny
	\item \textbf{Ciastocne rozhodnutelny} -- existuje TS, ktory zastavi pre $p \in L_{P}$, ale moze cyklit pre $p \not\in L_{P}$, teda $L_{P}$ je jazyk rekurzivne vycislitelne
\end{itemize}

\textbf{Jazyky typu 0}\\
Kazdy jazyk, ktory prijma TS, je jazyk typu 0
\begin{itemize}
	\item $G = (N,\Sigma,P,S)$
	\item $N = \{S\} \cup Q \cup (\Gamma \setminus \Sigma) \cup \{[,]\}$
	\begin{itemize}
		\item $S \to [q_{F}\Delta Y \Delta]$
		\item $\Delta] \to \Delta\Delta]$
		\item $xp \to qx$ ak $\delta(q,x) = (p,R)$
		\item $pxy \to xqy$ ak $\delta(q,x) = (p,L)$
		\item $py \to qx$ ak $\delta(q,x) = (p,y)$
		\item $\Delta\Delta] \to \Delta]$
		\item $[q_{0}\Delta \to \varepsilon$
		\item $\Delta] \to \varepsilon$
	\end{itemize}
\end{itemize}

Kazdy jazyk typu 0 je prijmany TS
\begin{itemize}
	\item Zostrojime dvojpaskovy NTS $M$
	\item Na prvu pasku zapiseme retazec $w$ a na druhu pociatocny neterminal $S$
	\item Nedeterministicky sa vyberie pravidlo $\alpha \to \beta \in P$ a aplikuje sa na druhej paske
\end{itemize}
$\mathcal{L}_{0} = \mathcal{L}_{TS}$

\textbf{Uzavretost voci operaciam}
\begin{itemize}
	\item Uzavretost voci $\cup$ -- zstrojime NTS $M_{L_{1} \cup L_{2}}$, ktory nedetermisticky vyberie, ci
		spusti povodny stroj $M_{1}$, alebo povodny stroj $M_{2}$
	\item Uzavretost voci $\cap$ -- zostrojime TS $M_{L_{1} \cap L_{2}}$, ktory spusti oba $M_{1}$ aj $M_{2}$
		a pokial obe prijmu, tak prijme aj $M_{L_{1} \cap L_{2}}$
	\item Uzavretost voci $\cdot$ -- zostrojime NTS $M_{L_{1}.L_{2}}$, ktory nedeterministicky rozdeli vstup $w$
		na 2 casti $w_{1}.w_{2}$ a spusti $M_{1}$ nad $w_{1}$ a $M_{2}$ nad $w_{2}$ a ak oba prijmu, tak prijme.
	\item Uzavretost voci $*$ -- zostrojime NTS $M_{L_{1}^{*}}$, ktory nedeterministicky vyberie prefix z retazca
		$w$ nad ktorym spusti $M_{L_{1}}$ a takto postupuje postupne nad zvyskom vstupu $w$. Prijme pokial $M_{L_{1}}$
		prijme kazdu cast.
	\item Rekurzivne jazyky su uzavrete voci komplementu -- dokazeme upravit uplny TS M tak, aby pre kazde nedefinovane
		$\delta$ presiel do $q_{reject}$, a taktiez umiestnit na lavy koniec pasky taky symbol, z ktoreho vedie len jeden
		prechod do $q_{reject}$. Pokial potom vymenime $q_{F}$ a $q_{reject}$ tak dostavame uplny TS $\overline{M}$
\end{itemize}

\textbf{Ak su $L$ a $\overline{L}$ rekurzivne vycislitelne, tak su oba rekurzivne}
\begin{itemize}
	\item Simulujeme $M$ a $\overline{M}$ krok po kroku na striedacku
	\item Prijmeme ak prijme $M$, odmietneme ak prijme $\overline{M}$
	\item Nakolko su rekurzivne jazyky uzavrete voci komplementu, tak $L$ aj $\overline{L}$ musia byt rekurzivne
\end{itemize}

\subsection*{Linearne Obmedzene Automaty}

LOA je NTS, ktory nikdy neopusti cast pasky, na ktorej je zapisany jeho vstup.\\
DLOA je DTS.\\
\textbf{Triedu jazykov generovanu kontextovymi gramatikami je mozne prijmat LOA}
\begin{itemize}
	\item $\mathcal{L}_{LOA} \subseteq \mathbb{L}_{1}$
	\begin{itemize}
		\item Ako u TS, len nebudeme mat pravidlo $\Delta] \to \Delta\Delta]$, ktore rozsiruje pasku a
			musime sa nejako postarat o nekontextove pravidla $[q_{0}\Delta \to \varepsilon$ a $\Delta] \to \varepsilon$.
	\end{itemize}
	\item $\mathcal{L}_{1} \subseteq \mathcal{L}_{LOA}$
	\begin{itemize}
		\item Ako u TS, ale zamedzime aby sa prekrocila hranica pasky
	\end{itemize}
\end{itemize}

\textbf{Kazdy kontextovy jazyk je rekurzivny}\\
Vzhladom na obmedzenie pasky moze dojst len ku konecnemu mnozstvu konfiguracii, cez ktore sa prejde na prijatie vstupu.
Dokazeme obmedzit zhora nejakou hranicou a odmietnut ak dojde k prekroceniu tejto hranice.\\
\\
\textbf{Nie kazdy rekurzivny jazyk je kontextovy}\\

\section{Rozhodnutelnost}

\subsection*{Jazyky mimo triedu 0}
\textbf{Pre kazdu abecedu existuje jazyk nad nou, ktory nie je typu 0}
\begin{itemize}
	\item Mnozina vsetkych TS nad $\Sigma$ je spocetna
	\item Mnozina vsetkych jazykov $2^{\Sigma^{*}}$ je nespocetna
		\begin{itemize}
			\item \emph{Dokaz}
			\item Predpokladajme ze $2^{\Sigma^{*}}$ je spocetna, tak existuje bijektivne zobrazenie $f: 2^{\Sigma^{*}} \leftrightarrow \mathbb{N}$
			\item Usporiadame reťazce $w_{i}$ za sebou. Teraz zvolíme jazyky $L_{i}$ pod seba a vždy dáme 1 do matice ak $w_{i} \in L_{j}$ a $0$ ak $w_{i} \not\in L_{j}$.
			\item Zostrojíme jazyk $L_{d} = \{ w_{i} \pipesep w_{i} \not\in L_{i}\}$. $L_{d}$ je ale odlišný od každého $L_{i}$ aspoň v 1 reťazci. Ale $L_{d} \in 2^{\Sigma^{*}}$,
				tým pádom ale $f$ nie je surjektivne.
		\end{itemize}
\end{itemize}

\subsection*{Problem zastavenia}
\textbf{Kodovanie TS}\\
Kodujeme stavy, znaky a prechody na postupnosti $0$ a $1$.\\
\textbf{Univerzalny TS}\\
Vstup $\langle M \rangle \# \langle w \rangle$ -- simulacia zakodovaneho TS $M$ nad zakodovanym vstupom $w$
\begin{itemize}
	\item 3 pasky
	\item 1. paska -- vstup
	\item 2. paska -- parcovna paska simulovaneho stroja
	\item 3. paska -- ulozeny stav a pozicia na paske simulovaneho stroja
\end{itemize}
\textbf{Problem zastavenia}\\
$HP = \{\langle M \rangle \# \langle w \rangle \pipesep M\text{ zastavi na }w\}$\\
Podla diagonalizacie nerozhodnutelny, a nechce sa mi tu pisat ten dokaz, studoval som to dost krat na to, aby som to vedel. \\

\textbf{Redukcia}
\begin{itemize}
	\item Redkujeme jeden problem $A$ na iny problem $B$ uplnym TS -- $A \le B$
	\item Nech $\Sigma_{1}, \Sigma_{2}$ su abecedy a $A$ jazyk nad $\Sigma_{1}$, $B$ jazyk nad $\Sigma_{2}$
	\item Reduckia dana zobrazenim $\sigma: \Sigma_{1}^{*} \to \Sigma_{2}^{*}$
	\item $\sigma$ zachovava prislusnost do jazyka $\forall w \in \Sigma_{1}^{*}: w \in A \Leftrightarrow \sigma(w) \in B$
	\item Plati nasledovne:
		\begin{itemize}
			\item Ak $A$ nie je rekurzivne vycislitelny, tak aj $B$ nie je rekurzivne vycislitelny
			\item Ak $A$ nie je rekurzivny problem, tak aj $B$ nie je rekurzivny problem
			\item Ak $B$ je rekurzivne vycislitelny, tak aj $A$ je rekurzivne vycislitelny
			\item Ak $B$ je rekurzivny, tak aj $A$ je rekurzivny
		\end{itemize}
\end{itemize}

\subsection*{Postov korespondencny problem}
\begin{itemize}
	\item \textbf{Postov system} -- je dany neprazdnym zoznamom $S$ dvojic neprazdnych retazcov nad $\Sigma$: $\langle (\alpha_{1}, \beta_{1}), ..., (\alpha_{n}, \beta_{n})\rangle$
		$\alpha,\beta \in \Sigma^{+}, n \ge 1$
	\item \textbf{Riesenie postovho systemu} -- neprazdna postupnost prirodzenych cisel $\langle i_{1},...,i_{k}\rangle$, taka ze
		\begin{equation*}
			\alpha_{i_{1}}\alpha_{i_{2}}...\alpha_{i_{k}} = \beta_{i_{1}}\beta_{i_{2}}...\beta_{i_{k}}
		\end{equation*}
	\item \textbf{Postov problem} -- Je postov system riesitelny?
	\item \textbf{PCP je nerozhodnutelny}
		\begin{itemize}
			\item Redukcia z $MP$
			\item $M_{\sigma}$ vytvori viacpaskovy TS $M'$, ktory bude mat v sebe zakodovany jeho vstup $\langle M \rangle \# \langle w \rangle$
			\item $M'$ bude mat minimalne 2 pasky, jednu pre $\alpha$ a jednu pre $\beta$ retazce Postovho systemu
			\item Najskor sa zapise na $\alpha$ pasku $\#$ a $\beta$ pasku $\#\langle\text{pociatocna konf. TS}\rangle$, co odpoveda dvojici $(\#,\#\langle...\rangle)$
			\item Na zaklade aktualnej konfiguracie na $\beta$ paske, vykonaj akciu budto vytvorenim dvojice $(qa,ap)$ pre posun doprava, dvojicu $(aqb,pab)$ pre posun dolava,
				dvojicu $(qa,pb)$ pre prepis pasky a inak zvysok konfiguracie z $\beta$ prekopiruj na $\alpha$.
			\item Ku koncu je treba obe pasky zarovnat a preto vytvaraj dvojice s $q_{F}$ so susednymi znakmi, dokedy nie su uz len aktualne konfiguracie zlozene s $q_{F}$ a $\#$
			\item Pokial existuje taka postupnost konfiguracii, tak je PCP riesitelny a zaroven $M$ prijma $w$
		\end{itemize}
\end{itemize}

\section{Vycislitelne funkcie}
\subsection*{Rekurzivne funkcie}
\textbf{Pociatocne funkcie}
\begin{itemize}
	\item Nulova funkcia -- $\xi() = 0$
	\item Funkcia nasledovnika -- $\sigma(x) = x + 1$
	\item Projekcia -- $\pi_{k}^{n}(x_{1},x_{2},...,x_{n}) = x_{k}$
\end{itemize}
\textbf{Primitivne rekurzivne funkcie}
\begin{itemize}
	\item 3 sposoby vytvarania novych funkcii
	\item Kombinacia -- $f \times g(\overline{x}) = (f(\overline{x}),g(\overline{x}))$
	\item Kompozicia -- $f \circ g(\overline{x}) = f(g(\overline{x}))$
	\item Primitivna rekurzia
		\begin{itemize}
			\item $f(\overline{x},0) = g(\overline{x})$
			\item $f(\overline{x},y+1) = h(\overline{x},y,f(\overline{x},y))$
		\end{itemize}
	\item Vsetky funkcie z triedy primitivne rekurzivnych funkcii mozu byt vytvorene uzitim tychto 3 technik na pociatocne funkcie
\end{itemize}
\textbf{Existuju funkcie, ktore su vycislitelne, ale nie su primitivne rekurzivne} -- vsetky striktne parcialne a aj niektore totalne (Ackermann)\\
Dokaz diagonalizaciou -- mozme vytvorit usporiadania funkcii $f_{1},f_{2},...$, nasledne vytvorime funkciu $f_{d} = f_{n}(n) + 1$ -- je totalna a vycislitelna, ale nie je
v zozname, lebo je odlisna od vsetkych $f_{n}$, pricom nie je primitivne vycislitelna\\
\textbf{Trieda totalnych vycislitelnych funkcii sa nazyva $\mu$-rekurzivne funkcie}\\
\\
\textbf{Parcialne rekurzivne funkcie}
\begin{itemize}
	\item Minimalizacia
	\item $g(\overline{x},y) = 0$
	\item $g(\overline{x},z)$ je definovane $\forall z < y: z \in \mathbb{N}$
	\item Inak zapisujeme
		\begin{equation*}
			f(\overline{x}) = \mu y[g(\overline{x},y) = 0]
		\end{equation*}
	\item Trieda parcialne rekurzivnych funkcii vznikne bud kombinaciou, kompoziciou, primitivnou rekurziou a minimalizaciou
\end{itemize}

\subsection*{Turingovsky vycislitelne funkcie}
\begin{itemize}
	\item TS $M$ vycisluje (pocita) parcialnu funkciu $f: \Sigma_{1}^{*m} \to \Sigma_{2}^{*n}$, ked pre kazdu
		$(w_{1},w_{2},...w_{m})$ odpovedajucu konfiguracii pasky $\underline{\Delta}w_{1}\Delta w_{2}\Delta...\Delta w_{m}\Delta^{\omega}$
		budto
		\begin{itemize}
			\item Zastavi s konfiguraciou $\underline{\Delta}v_{1}\Delta v_{2}\Delta...\Delta v_{n}\Delta^{\omega}$,
				pokial $f(w_{1},w_{2},...,w_{m}) = (v_{1},v_{2},...,v_{n})$
			\item Cykli, ked $f(w_{1},w_{2},...,w_{m})$ nie je definovane
		\end{itemize}
	\item \textbf{Parcialna funkcia, ktoru je mozne vycislit Turingovym strojom sa nazyva Turingovsky vycislitelna}
\end{itemize}
\textbf{Turingovska vycislitelnost parcialne rekurzivnych funkcii}
\begin{itemize}
	\item Uvazujeme TS nad $\{0,1\}$
	\item Najskor TS, ktore vycisluju pociatocne funkcie $\xi,\sigma,\pi$
		\begin{itemize}
			\item $\xi$ -- $R0L$
			\item $\sigma$ -- binarny inkrement, rozsirenie pasky
			\item $\pi$ -- posun $k$-teho vstupu na zaciatok, prepisanie $k+1$-eho az $n$-teho vstupu
		\end{itemize}
	\item Kombinacia -- 3-paskovy TS, ktory na 2. paske vycisli $g_{1}$, na 3. paske vycisli $g_{3}$
		a potom prekopiruje obsah tychto pasok na 1. pasku a oddeli pomcou $\Delta$
	\item Kompozicia -- Vycisli $g_{1}$ a nasledne vycisli $g_{2}$
	\item Primitivna rekurzia
		\begin{itemize}
			\item Na pasku zapisujeme postupne
				$\Delta\overline{x}\Delta y\Delta\overline{x}\Delta y-1\Delta...\Delta\overline{x}\Delta0\Delta\overline{x}\Delta^{\omega}$
			\item Hlavu posunieme na $\Delta$ pred poslednym $\overline{x}$ a spustime simulaciu $M_{g}$, ktory vycisluje $g$
			\item Hlavu potom znova posunieme pred posledne $\overline{x}$ a spustame simulaciu $M_{h}$, ktory vycisluje $h$
			\item Na konci mame vycislene $h(\overline{x},y,f(\overline{x},y))$ co odpoveda $f(\overline{x},y+1)$
		\end{itemize}
	\item Minimalizacia
		\begin{itemize}
			\item 3-paskovy TS
			\item 1.paska -- vstup $\overline{x}$
			\item 2.paska -- hladane minimalne $y$
			\item 3.paska -- pracovna paska
			\item Na 2. pasku zapis hodnotu $0$
			\item Prekopiruje sa na 3. pasku obash 1. a 2. pasky a spusti sa $M_{g}$ na 3. paske
			\item Ak sa $g$ vycisli na $0$, tak algoritmus prepise 2. pasku na 1. pasku, inak zvysi obsah 2. pasky o $1$
				a pokracuje sa v simulacii na 3. paske
		\end{itemize}
\end{itemize}

\textbf{Kazdy vypocetny proces vykonavany Turingovym strojom je vycislenim nejakej parcialne rekurzivnej funkcie}\\
Dokaz je vytvorenie mnozstva funkcii ktore reprezentuju simulaciu TS

\section{Zlozitost}
Zlozitost (ciastocne) rozhodnutelnych problemov a ich rozhodovania
\begin{itemize}
	\item Mozeme rozlisit
	\item Zlozitost najhorsieho pripadu
	\item Zlozitost najlepsieho pripadu
	\item Zlozitost priemerneho pripadu
\end{itemize}
\subsection*{Zlozitost vypoctu TS}
\begin{itemize}
	\item \textbf{Casova zlozitost} -- pocet prechodov TS, ktore sa musia vykonat
	\item \textbf{Priestorova zlozitost} -- pocet buniek pasky TS, ktore su pozadovane na vypocet
	\item \textbf{Pokial je casova zlozitost $n$, tak priestorova zlozitost nie je vacsia ako $n+1$}
	\item $k$-paskovy DTS (NTS) prijma jazyk $L$ v case $T_{M}: \mathbb{N} \to \mathbb{N}$, pokial
		$L = L(M)$ a $\forall w \in L$ prijme (moze prijat) v maximalne $T_{M}(|w|)$ krokoch
	\item $k$-paskovy DTS (NTS) prijma jazyk $L$ v priestore $S_{M}: \mathbb{N} \to \mathbb{N}$, pokial
		$L = L(M)$ a $\forall w \in L$ prijme (moze prijat) s pouzitim maximalne $S_{M}(|w|)$ buniek pasky
\end{itemize}
\subsection*{Zlozitost vypoctu mimo prostredie TS}
\begin{itemize}
	\item V inom prostredi ako TS nemusi mat kazda operacia rovnaku cenu
\end{itemize}
\subsection*{Asymptoticke obmedzenie zlozitosti}
\begin{itemize}
	\item Nech $\mathcal{F}$ je mnozina funkcii $f: \mathbb{N} \to \mathbb{N}$, pre danu funkciu $f$ definujeme mnoziny
		$O(f(n)), \Theta(f(n))$ a $\Omega(f(n))$
	\item $O(f(n))$ -- \textbf{Asymptoticke horne obmedzenie funkcie $f(n)$}
		\begin{itemize}
			\item $O(f(n)) = \{g(n) \in \mathcal{F} \pipesep \exists c \in R^{+}:\exists n_{0} \in \mathbb{N}: \forall n\ge n_{0}: g(n) \le c.f(n)\}$
		\end{itemize}
	\item $\Omega(f(n))$ -- \textbf{Asymptoticke dolne obmedzenie funkcie $f(n)$}
		\begin{itemize}
			\item $\Omega(f(n)) = \{g(n) \in \mathcal{F} \pipesep \exists c \in R^{+}:\exists n_{0} \in \mathbb{N}: \forall n\ge n_{0}: c.f(n) \le g(n)\}$
		\end{itemize}
	\item $\Theta(f(n))$ -- \textbf{Asymptoticke obojstranne obmedzenie funkcie $f(n)$}
		\begin{itemize}
			\item $\Theta(f(n)) = \{g(n) \in \mathcal{F} \pipesep \exists c_{1},c_{2} \in R^{+}:\exists n_{0} \in \mathbb{N}: \forall n\ge n_{0}: c_{1}.f(n) \le g(n) \le c_{2}.f(n)\}$
		\end{itemize}
\end{itemize}
\subsection*{Zlozitost problemov}
\begin{itemize}
	\item Majme dane funkcie $t,s: \mathbb{N} \to \mathbb{N}$ a nech $T_{M}$, resp. $S_{M}$ znaci casovu resp. priestorovu zlozitost TS $M$
	\item Definujeme nasledujuce triedy zlozitosti deterministickych a nedeterministickych TS
		\begin{itemize}
			\item \emph{DTime}$[t(n)] = \{L \pipesep \exists k-\text{paskovy DTS }M: L = L(M) \land T_{M} \in O(t(n))\}$
			\item \emph{NTime}$[t(n)] = \{L \pipesep \exists k-\text{paskovy NTS }M: L = L(M) \land T_{M} \in O(t(n))\}$
			\item \emph{DSpace}$[s(n)] = \{L \pipesep \exists k-\text{paskovy DTS }M: L = L(M) \land S_{M} \in O(s(n))\}$
			\item \emph{NSpace}$[s(n)] = \{L \pipesep \exists k-\text{paskovy NTS }M: L = L(M) \land S_{M} \in O(s(n))\}$
		\end{itemize}
\end{itemize}
\textbf{Bezne triedy zlozitosti problemov}\\
\begin{itemize}
	\item $P = \bigcup DTime[n^{k}]$
	\item $NP = \bigcup NTime[n^{k}]$
	\item $PSPACE = \bigcup DSpace[n^{k}]$
	\item $NSPACE = \bigcup NSpace[n^{k}]$
	\item $EXP = \bigcup DTime[2^{n^{k}}]$
	\item $NEXP = \bigcup NTime[2^{n^{k}}]$
	\item $EXPSPACE = \bigcup DSpace[2^{n^{k}}]$
	\item $NEXPSPACE = \bigcup NSpace[2^{n^{k}}]$
\end{itemize}


\end{document}

